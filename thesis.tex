\documentclass[12pt]{article}

\usepackage{graphics}
\usepackage{epsfig}
\usepackage{times}
\usepackage{amsmath}
\usepackage{setspace}
\usepackage{apacite}

\topmargin      0.0in
\headheight     0.0in
\headsep        0.0in
\oddsidemargin  0.0in
\evensidemargin 0.0in
\textheight     9.0in
\textwidth      6.5in

\title{{\bf Information, Legitimacy and the Emergence of Industrial Markets} \\ 
\it Thesis proposal}
\author{ {\bf Jameson K. M. Watts}  \\
Eller College of Management \\
University of Arizona\\
{\small jamesonw@email.arizona.edu}
}
\date{\today}

\newtheorem{prop}{Proposition}
\newtheorem{hypo}{Hypothesis}

\doublespacing
\begin{document}

\pagestyle{plain}
\pagenumbering{roman}
\maketitle

%\pagebreak
%\begin{abstract}
%\end{abstract}

%\pagebreak
%\tableofcontents
%\pagebreak

\cleardoublepage
\pagenumbering{arabic}

\section{Introduction}
\label{ch:intro}

When a firm like Genentech--one of the world's leading biotechnology firms--chooses a new alliance partner, it does so despite substantial uncertainty regarding the long-term potential of the relationship. A young firm with several accomplishments in an established area of research, may still under-perform in an alliance context. In this case, Genentech may seek advice from members of its partner network regarding the ability of the young firm to collaborate \cite{492}. If, on the other hand, the firm is working on new or untested therapies, then Genentech lacks a frame of reference upon which to base its evaluation. In this case, Genentech may assess potential based on what it knows about the young firm's other partners--a tie to a prestigious University or research lab, can be a signal of quality \cite{496, 473, 684}.

These scenarios are instructive of why social structure matters in alliance choice \cite<e.g.>{866} and more generally, why it matters for many of the strategic decisions that involve longer term exchange arrangements \cite{875, 876}. In the marketing literature, the nature of inter-firm relationships was famously addressed by \citeA{864} in his seminal article on the function of ``domesticated markets." Subsequent research focused on relationships at the dyadic level, whether buyer and supplier \cite{893}, service providers and clients \cite{895}, or manufacturers and distributors \cite{899}. However, research on inter-firm relationships has expanded in recent years to consider the influence of constructs that operate at the network level of analysis \cite{874, 469, 815}. 

Progress in the area of inter-firm networks is especially evident in research that focuses on technology-intensive (TI) industries like software and biotechnology \cite{867, 469, 775}. Firms in TI industries are known to form relationships for a variety of strategic reasons including acquisition of needed resources \cite{878}, access to diverse information \cite{512} and as a way to diversify exposure to risk \cite{775}. However, TI industries are characterized by an expanding knowledge base, which pressures firms to regularly reevaluate the deployment of their resources in the marketplace. When firms frequently change the way they operate, industry standards are transient and it becomes difficult to evaluate the potential of a new relationship based on the behavior of comparable firms. Instead, strategic considerations are typically complemented by a secondary source of information accessed through existing relationships or inferred based on the reputation of a firm's relationship portfolio \cite{496, 684}. Yet, despite this concurrence of the strategic and the social, there are relatively few studies that incorporate arguments from both perspectives.

What we do know about the joint influence of social and strategic influence on inter-firm relationship formation, comes largely from the management and sociology literatures.\footnote{Notably, the use of theory and concepts from sociology and firm theory to explain ``domesticated markets" was predicted by \citeA{864}.} For example, \citeA{812} show that entrepreneurial firms forge new relationships based on a combination of strategic need and social opportunity. \citeA{866} generalizes this finding to established firms by using measurements of technical, commercial and social capital. \citeA{513} go a step further to show that relationship formation is both the product of extant social structure and a response to changes in strategic context. Finally, \citeA{900} show that the balance between need and opportunity can be affected by a firm's reputation.

Although these studies have increased our understanding of when social and strategic factors jointly influence relationship formation, the nature of this interdependence remains underdeveloped. One notable limitation is the way in which resources are defined. In general, researchers make assumptions about the value of a firm's resource profile in isolation, when in fact the value of a resource depends crucially on its deployability in an industrial market with {\it many} actors. These other actors must acknowledge the resource's proposed value or it can not be deployed \cite<cf.>{134, 188, 877}. It follows, that consensus regarding the proper use of a resource is implicit in strict definitions of its value. And yet, this assumption can be problematic when employed to explain behavior in rapidly developing industries with multifaceted resources. For example, the value of a genetic patent varies in accordance with the other patents with which it is combined to produce a medicine. Similarly, a firm's human capital can be deployed (and redeployed) in a multitude of strategic configurations throughout an industry. Under this logic, it is not a particular resource that has value, but rather the strategy of deploying the resource--the industry practice--which defines its value {\it in use} \cite{881}. 

Scholarship, concerned with the formation of inter-firm relationships is lacking in theory that can address both the availability \emph{and} the deployability of resources in industrial markets. This apparent gap in our knowledge is the primary motivation for the current work. Building on work in economic sociology focused on legitimacy \cite{826, 877, 824, 493} and network research focused on the conditional nature of embeddedness \cite{872, 869, 871, 514}, I offer a theoretical framework that conceptualizes resource deployability as a key antecedent to the formation of inter-firm relationships. The empirical setting is the field of Biotechnology between 1991 and 2004. This period represents a time of significant industry growth in which the logic of production (and concomitant strategies) underwent several significant changes. Relationship choices are captured by a database that covers all formal contractual arrangements between biotech firms. It contains data on both the timing of the alliance and attributes specific to each relationship. Information about industry practices is captured by a database of publications from a prominent biotech trade journal and is comprised of more than 20,000 articles.

\section{A Behavioral Account of Legitimacy}
\label{ch:prop}
\subsection{The Legitimacy Construct}
\begin{small}
\begin{quote}
``\ldots legitimacy depends on the consensus among agents (audiences) that the features and activities of organizations (candidates) are appropriate and desirable within a widespread, taken-for-granted system of norms or social codes \ldots terms we use synonymously to indicate cultural phenomena that prescribe and proscribe behavior in specific circumstances." \cite[p. ~146]{825}
\end{quote} 
\end{small}

In the social sciences, the construct of legitimacy is typically identified in relation to some social object. A brief reading of the literature suggests that legitimacy can be attained by institutions, professions, categories and persons. Examples include the United States legal system, the medical profession, SIC industry classifications and children \cite{493, 890, 891, 885}. However, this interpretation presumes that legitimacy is something that can be \emph{possessed}. As such, researchers have used the construct in a manner similar to other \emph{possess-able} constructs like financial and social capital. Yet, such a broad reading of the definition can create ambiguities when studied in parallel with other commonly referenced constructs like status, prestige and reputation. Is it Genentech's status or its legitimacy that makes it such a sought-after alliance partner? 

A more narrow interpretation--and the one I will adopt here--defines legitimacy as a property of behavior that conforms to a formal or informal set of rules. Under this definition, legitimacy is \emph{performed} rather than possessed. Rules are assumed exogenous, and the legitimacy of some behavior is judged in relation to its degree of conformity to those rules. For instance, a physician cannot possess legitimacy \emph{per se}; however her \emph{behavior} is legitimate when it conforms to the standards put forth by her profession. Similarly, a child is neither legitimate nor illegitimate, rather the behavior that led to its conception either conforms, or does not conform to a set of rules for procreation. 

A performance-based interpretation of legitimacy entails the presence of an audience \cite{825, 904, 877}. An analysis of behavior that occurs absent immediate or potential scrutiny by a group of observers is outside the purview of this thesis. However, given some meaningful audience, actors typically choose legitimate (conforming) behavior in order to avoid sanctions \cite{493}. For instance, male job-seekers often wear a suit and tie to interviews. Their outfit is unlikely to secure them a job, but alternative forms of dress could very well hinder their chances \cite{905}. In other words, legitimate behavior is primarily motivated by a desire to avoid negative outcomes rather than by the expectation of positive gains. Granted, some would argue that engaging in legitimate behavior provides subsequent access to environmental resources \cite<see e.g.>{877}. However, I tend to follow the logic of \citeA{905}, and view legitimate behavior as more or less the entry fee in a competitive game--conformity may grant you access, but it doesn't necessarily affect your chances of winning.

Variation in the legitimacy of some behavior is affected by variation in the consensus view of an audience \cite{825}. If consensus is high, then the level of legitimacy is the extent to which a behavior conforms to the consensus view.  However, it is often the case that audience members do not agree on what constitutes appropriate behavior. For instance, in many populations there is disagreement over the rules that govern legitimate procreation. In this case, the measured level of conformity may be the same (on average), but its variance will be higher. If consensus regarding the set of rules which define conformity is low, then behaviors will have less legitimacy even if they conform to the average opinion of an audience. 

In sum, we can think of the legitimacy construct as comprised of two primary dimensions: morality and consensus. The morality dimension is determined by a behavior's conformity to a predetermined set of rules. The consensus dimension is determined by the level of agreement about those rules. Once agreement is established, the level of conformity can be more easily judged. It is therefore vital that we are clear on the meaning of consensus before proceeding.


\subsection{Legitimacy and Social Structure}

\begin{small}
\begin{quote}
``...perhaps we should follow Simon's lead and look more closely at the process of decision making. Instead of beginning our analysis by assuming a set of options we should inquire what options come to be included in the choice set--for, as Alchian recognized, the optimum cannot be chosen unless it has been thought of." (Loasby 1999 p. 31)
\end{quote}
\end{small}

If legitimacy is performance, then social structure sets the stage. At the macro scale, institutions, status hierarchies and categorical or professional memberships often define the set of rules against which legitimacy is evaluated. At the meso scale, social networks can constitute the audience passing judgement on a behavior. Indeed, the meso and macro are often reflective of one another \cite<cf.>{889}. The configuration of social ties between actors commonly resembles higher-order structures (like group membership) in which consensus regarding appropriate behavior is high \cite{902}. Similarly, certain configurations of social ties can facilitate consensus, and thus reinforce the boundaries of a group \cite{903}.

Legitimacy facilitates reflection in social structure when new tie formation is considered a behavior subject to the evaluation of an audience. If the legitimacy of a new tie is evident, then tie formation can proceed based on other factors. However, if legitimacy is low, audience reaction to the proposed tie is less certain and strategic actors may seek out alternative sources of information before making their decision. From the literature on status \cite<e.g.>{514} and social capital \cite<e.g.>{872}, we know that existing social ties are a viable source for this type of information. For instance, \citeA{872} showed that women were promoted faster when they attached themselves to an older male sponsor. At the time, there was little consensus regarding the role of women in top management, so the promotion of women was not considered legitimate behavior. For qualified females, the tie to an older male sponsor removed some of the uncertainty from the promotion decision.  

The use of extant social ties to reduce uncertainty is a frequently referenced explanation for new tie formation \cite{492}. However, there are two different views on how this occurs. The first conceptualizes social ties as conduits for the flow of information \cite<e.g.>{568}. In this view, new ties are more likely to originate in an actor's extended network. This occurs because information about tie candidates can flow across linkages that are already in place. In network parlance, this tendency for ties of ties to themselves form a tie, is referred to as transitive (or triadic) closure and can result in social networks characterized by high degrees of local clustering \cite{907}.

Social ties can also reduce uncertainty by signaling unobservable quality \cite{514}. For instance, when an actor has many direct ties, or has ties to other high status actors, their quality is inferred based on the assumption of good judgement by those that are already connected. For instance, individuals often infer the quality of a restaurant by observing how many patrons are currently dining there. Since, the existence of ties promotes the formation of even more ties, this process is commonly referred to as accumulative advantage and can lead to social networks characterized by large disparities in the amount of ties each actor is able to cultivate \cite{538}. 

A disparate tie distribution and local clustering are two of the most frequently observed properties of social networks \cite{907}. Yet, the proposed mechanisms that lead to these features--reliance on extant social ties for supplemental information--is the result of uncertainty. This implies that a \emph{reduction} in uncertainty can moderate the influence of existing ties on choice. A few researchers have begun to grapple with this trade-off. For instance, \citeA{800} showed that firms sought out new ties when facing individual-level uncertainty, but doubled down on existing ties when facing uncertainty in the environment. In a similar theoretical account, \citeA{871} showed that the borrowing behavior of large U.S. firms became less dependent on director interlocks as the role of financial executive gained prominence within firms. 

To the extent that legitimacy facilitates certainty regarding the outcome of a decision, then as tie formation in a given context becomes more legitimate, new ties can be formed without relying on existing ties for additional information. In other words, the tendency of networks to evolve according to accumulative advantage and transitive closure should diminish as the legitimacy of tie formation increases. This is because alternative (lower status and less connected) choices from the same pool of candidates become less risky. This statement can be framed more formally as a set of propositions:

\begin{prop}
\begin{minipage}[t]{4.5 in}
The more (less) legitimate the formation of ties that define a social network, the more (less) effect that transitive closure will have on that network's evolution.
\end{minipage}
\end{prop}

\begin{prop}
\begin{minipage}[t]{4.5 in}
The more (less) legitimate the formation of ties that define a social network, the more (less) effect that accumulative advantage will have on that network's evolution.
\end{minipage}
\end{prop}


\clearpage

\bibliographystyle{apacite}
\bibliography{cites.bib}

\end{document}
