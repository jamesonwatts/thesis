\chapter{Final Thoughts \label{final}}


\begin{small}
\begin{quote}
``...perhaps we should follow Simon's lead and look more closely at the process of decision making. Instead of beginning our analysis by assuming a set of options we should enquire what options come to be included in the choice set--for, as Alchian recognized, the optimum cannot be chosen unless it has been thought of" (Loasby 1999, p. 31). 
\end{quote}
\end{small}

As Loasby in the above quote recognizes, decisions in the marketplace and the ensuing process of exchange is, and always has been, linked to the finite knowledge and understanding of market actors. This occurs because individuals have limited cognitive resources and are heterogeneous in the use of these resources (Simon 1957). In Chapter \ref{law}, I introduced a concept called market entropy as a way to describe the degree of this knowledge heterogeneity in a market. In low entropy, alpha markets, we can reliably distinguish between the representations of various actors selected at random. In high entropy beta markets, most actors share similar representations.

I proposed that in alpha markets, exchange would follow an increase in representational homogeny. This proposition was supported in Chapter \ref{lang} by showing that the trading volume of biotech firms increased over the long term with increases in the consistency of language. However, the time-series were cointegrated, such that language consistency responded to increased trading volume as well. The codependency of these two constructs has intuitive appeal.  Just as greater shared understanding can promote exchange, the exchange process itself can result in more shared understanding.

I also found that short-term decreases in language consistency (i.e. new information) was predictive of transient increases in trading volume. This supported the second proposition of Chapter \ref{law} in which I suggested that reductions in entropy would prompt exchange when markets had reached a state of relative equilibrium. Whereas the effect of consistency was long-term and trending, the effect of inconsistency was short-term and stationary.

Chapter \ref{signal} asked how the knowledge heterogeneity concept developed in Chapter \ref{law} and operationalized in Chapter \ref{lang} would affect the performance of market actors. By integrating knowledge-based theories of the firm (Powell et. al., 1996) and the work of Podolny (2005) on structural market signals, I offered a description of performance based on both the substantive and symbolic advantages of network position. The uncertainty caused by high levels of knowledge heterogeneity positively moderated the symbolic advantage of network position but negatively moderated the substantive advantage.

Taken together my findings have some practical implications for managers. Foremost, shared understanding increases exchange over the long term. Moreover, the construct is accessible via analysis of written language. Thus, tracking this measure could be a useful forecasting metric. For instance, managers interested in the volume of activity within their industry may consider use of language consistency as a leading indicator. A natural extension of this finding is to look at less formal text like that generated on product review websites or Twitter to see if user-generated content has the same predictive capacity. This would also allow for measures to be delivered to managers as the text naturally occurs in real time.

Not only does shared understanding promote exchange, it differentially \emph{directs} exchange. When language uncertainty was low (i.e. high degree of shared understanding), exchange was directed at less central market actors. When uncertainty was high, exchange was concentrated towards an industry's central actors; however, the fruits of this exchange were more difficult for the central actors to use productively. 

Entrepreneurs may have a difficult time securing the resources necessary to survive in an industry characterized by high language uncertainty. However, they can compete on more equal footing with the central actors in the use of the resources they do have. This is because the central actors are unable to fully capitalize on the information flowing across network ties. In these cases of high uncertainty, entrepreneurial firms that have sufficient resources (e.g. financing), may outperform their peers.

In contrast, new resources are distributed much more broadly under low language uncertainty (i.e. high consistency). In this case, central firms should continue to use their networks to gain an information advantage and entrepreneurial firms should focus on forming alliances of their own that provide access to similar information.

While there is still much to be done in the future, the goal of the current work has been to provide a unified study of the knowledge-exchange relationship. I approached the the problem using theoretical arguments in Chapter \ref{law}, descriptive and aggregate causal inference in Chapter \ref{lang} and specific hypothesis testing at the firm-level in Chapter \ref{signal}. Each of these chapters built upon each other in what I hope stands as a significant contribution to our understanding of this complex issue.

