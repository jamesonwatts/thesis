\chapter{Introduction\label{intro}}

Markets are fluid. Over time, the dominant designs, processes and paradigms that define an industry invariably succumb to productive innovation or changes in fashion (Arthur 2009; Schumpter 1934; Simmel 1957). Take for example the recent upheaval of the cell phone market following Apple's release of the iPhone. When it was introduced in 2007, one could clearly differentiate Apple's product from all others; however, subsequent imitation of the iPhone produced a market in which nearly all cell phones look (and perform) alike. The iPhone was a harbinger of the new dominant design.

These cycles of innovation and fashion are not limited to consumer markets. Business markets (often defined by longer term inter-firm relationships) are subject to similar transformations. For example, current practices in the biotechnology industry are quite distinct from those accompanying its emergence from university labs in the second half of the 20th century (Powell et al. 2005). Technologies that were once viewed as radical have undergone a process of legitimation and integration into mainstream healthcare delivery systems. Practices that were dominant in the 1980's gave way to newer business models in the 1990's and feedback from downstream providers changed the way drugs were delivered to patients (Wolfe 2001). 

During periods of transition, market actors face great difficultly anticipating reactions to their behavior (practices, products, etc.). How they deal with this uncertainty is an interminable source of academic inquiry in the social sciences (see e.g. Anderson 1965; Simon 1957; Thompson, 1967) and, in a broad sense, it is the primary concern of the current work as well. However, I am focused specifically on the turmoil caused by transitions in technology, taste and attention over time--the disagreements which occur as market actors collectively shift their practices from one paradigm to the next (Colyvas and Powell 2008). If innovations are assumed to arise locally and diffuse gradually (see e.g. Bass 1965; Rogers 2002), then transient differences in knowledge are a natural outcome. Those closest to, or most interested in an innovation will have greater knowledge than those furthest away or less involved. Thus, for a period following some shift in technology, taste or attention, market participants will vary in their knowledge and interpretation of the change.

In the following chapters, I investigate the ramifications of this sort of knowledge heterogeneity on the exchange behavior and subsequent performance of market participants. In a general sense, chapters build upon each other in a progression from abstract, to descriptive to specific tests of theory. However, each can also stand by itself as an independent examination of the knowledge-exchange relationship. 

In Chapter \ref{law}, I review the history and development of Alderson's (1965) `law of exchange' in the marketing literature and propose an extension based on insights from information theory. A concept called market entropy is introduced to describe the distribution of knowledge in a field and propositions are offered to explain the exchange behavior expected when this distribution changes. I conclude with a discussion of how the proposed extension can help scholars and practitioners think about markets and marketing in a new and useful manner. 

Chapter \ref{lang} investigates knowledge heterogeneity through its relation with written language. Drawing on social-constructionist theories of classification (Goldberg 2013) and insights from research on the legitimation process (Colyvas and Powell 2008), I argue for a measure of field-level consensus based on changes in the frequency distribution of descriptive words over time. This measure is operationalized using fourteen years of trade journal articles from the biotech industry and is shown to support in a general sense, the propositions offered in Chapter \ref{law}. 

Chapter \ref{networkl} builds on the arguments and evidence developed in Chapters \ref{law} and \ref{lang} to test theory on the structural advantages of a firm's position in a network of strategic alliances. Prior work has documented returns to network centrality based on the premise that central firms have greater and more timely access to information about industry developments (Koput et al. 1998; Powell et al. 1999). However, other research claims that benefits to centrality accrue based on the signal that such a position provides about an actor's underlying quality (Malter 2014; Podolny 1993; 2005). I investigate this tension in the literature and offer new insights based on the moderating effect of the measure developed in Chapter \ref{lang}.

%Some deep and insightful summary of my contributions.