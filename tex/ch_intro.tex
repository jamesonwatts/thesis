\chapter{Introduction\label{intro}}

Markets are fluid. Over time, the dominant designs, processes and paradigms that define an industry invariably succumb to productive innovation or changes in fashion (Arthur 2009; Schumpter 1934; Simmel 1957). Take for example the recent upheaval of the cell phone market following Apple's release of the iPhone. When it was introduced in 2007, one could clearly differentiate Apple's product from all others; however, subsequent imitation of the iPhone produced a market in which nearly all cell phones look (and perform) alike. The iPhone was a harbinger of the new dominant design.

These cycles of innovation and fashion are not limited to consumer markets. Business markets (often defined by longer term inter-firm relationships) are subject to similar transformations. For example, current practices in the biotechnology industry are quite distinct from those accompanying its emergence from university labs in the second half of the 20th century (Powell et al. 2005). Technologies that were once viewed as radical have undergone a process of legitimation and integration into mainstream healthcare delivery systems. Throughout this process, the way in which potential therapies are financed, developed and ultimately commercialized, changed significantly. Practices that were dominant in the 1980's gave way to newer business models in the 1990's and feedback from downstream providers changed the way in which drugs were delivered to patients (Wolfe 2001). 

Scholars across the social sciences have long been interested in how market actors react to uncertainty in their environment (Anderson 1965; Simon 1957; Thompson, 1967) and indeed, that is the primary concern of the current work as well. However, I am focused specifically on the uncertainty caused by transitions in technology, taste and attention over time--the uncertainty which occurs as market actors collectively shift their practices from one paradigm to the next. This is a form of environmental uncertainty, and as such, it conforms in a basic sense to Pfeffer and Salancik's (1978) definition, which describes uncertainty as  ``the extent to which future states of the world cannot be anticipated and accurately predicted." 

But which future states `cannot be anticipated?' Drawing on social constructionist theories of market acceptance (e.g. Colyvas 2007; Rosa et al., 1999; Zajac and Westphal 2004), I propose that during transitions of technology, taste and attention, it becomes more difficult to evaluate the quality of market participants. This occurs because quality assessments are based on an evolving set of standards. Thus, the uncertainty of highest consequence stems from an inability to anticipate the future standards of quality. How then, does this type of uncertainty affect the behavior of market participants? If evolving standards decrease the capacity of market actors to reliably evaluate quality, then I propose exchange behavior will be affected by an increased reliance on structural signals of market value (Podolny 1994; Malter 2014). These two arguments are elaborated in the following sections.

\section{Uncertainty, Legitimacy and Language}

\begin{small}
\begin{quote}
``Grammars in both cooking and engineering exist not just as rules but as a set of unspoken practices taken for granted" (Arthur 2009 p. 77)
\end{quote}
\end{small}

Capturing the effects of environmental uncertainty over time is no easy task. Indeed, much of the noteworthy empirical work in this area defines uncertainty according to some exogenously determined classification, and often one that is fixed in time. For instance, Podolny (1994) examines exchange relationships in the issuance of investment-grade vs. non-investment-grade debt. Firms operating in the latter context are assumed to experience higher levels of uncertainty. In a more contemporary rendering, Erdem et al. (2006) use a country's predetermined score on an uncertainty avoidance scale to make claims about how individuals in that country react to signals of brand credibility. 

Studies that attempt to track environmental uncertainty over time, often do so using measures of volatility in a portfolio of stocks (see e.g. Beckman et al. 2004). However, this might be better understood as systematic risk (Thomaz and Swaminathan 2014). Under risk, the probabilities associated with future outcomes are assumed to be known, whereas uncertain prospects have unknown probabilities (Tversky and Fox 1995). More importantly, the stock market represents a limited sample of market participants and likely misses important behavior undertaken by smaller or newer participants in a field. 

Other studies have looked at the percentage change in an industry's patenting activity as a way to capture technical uncertainty (see e.g. Goerzen 2007; Luque 2002). However, technical uncertainty is only one dimension of environmental uncertainty and patents are only one facet of market behavior. Moreover, patenting activity is itself subject to changes in practice over time. For example, biotech firms have learned to over-apply for patents as a way to conceal the innovations they are actually using (Wolfe 2001).

A parallel stream of research has examined the related concept of legitimacy, which can be defined as ``consensus among agents (audiences) that the features and activities of [market actors] are appropriate and desirable within a widespread, taken-for-granted system of norms or social codes" (Cattani et al. 2008, p. 147). Legitimacy is linked to uncertainty through the legitimation process. Innovations in behavior arise locally and spread gradually to others in a field (Colyvas 2007). As this process unfolds, there are periods of high disagreement about which behaviors are appropriate (or optimal). This in turn, increases the uncertainty actors have about market reactions to future behavior.

This formulation of environmental uncertainty--derived from consideration of the legitimation process--comports nicely with my focus on field-level transitions in technology, taste and attention. However, a brief review of the empirical literature on legitimacy raises an additional concern: the behavior deemed legitimate is typically defined ex post--that is, operationalizations are based on the present incarnation of a category and then traced to some earlier period. For example, Petkova et al., (2014) examine the legitimacy of investing in the clean energy sector by counting historical references to the terms ``clean energy," ``green energy" and ``alternative energy" in media articles. In practice, categories emerge organically (Goldberg 2013) and may cycle through several incarnations of appropriate behavior before they settle on the perspicuously demarcated boundaries that appear to the present observer. More importantly, this formulation excludes the struggle between competing paradigms that characterizes the typical legitimation process (Colyvas and Powell 2007).

How then, do we attempt to understand the uncertainty caused by transitions in technology, taste and attention in an unbiased manner? Despite it's shortcomings, the literature on legitimacy provides several useful insights. Foremost, the frequency of  behavior is a meaningful yardstick of general acceptance (Hannan and Carroll, 1992). Second, the frequencies of language used in the media are representative of an unfolding legitimation process (Petkova et al., 2014; Rosa et al., 1999). Using these insights, I propose a general notion of environmental uncertainty based on changes in the frequency of language use. This can be accomplished without imposing ex post classifications by tracking language frequencies as they naturally occur--that is, the frequency distribution of descriptive words at various points in time. Changes in a frequency distribution from one period to the next is thus assumed to represent shifts in the importance of behavior described by the language. The volatility of these changes can capture disagreement about importance and hence uncertainty as previously described. 

\section{Market Signals and Exchange}

\begin{small}
\begin{quote}
``The heterogeneous market is cleared by information" (Alderson 1965, p. 348)
\end{quote}
\end{small}

To what extent does environmental uncertainty affect the exchange behavior of market participants?  As I argued above, uncertainty results from disagreement over the importance (or appropriateness) of market behaviors. This will naturally occur if innovations (in practice, technology and fashion) arise locally and diffuse gradually (Colyvas 2007; Rogers 2002). For a period following some innovation, market participants will vary in knowledge and opinion of that innovation. If uncertainty is capturing this knowledge heterogeneity in the field, I propose that variation in uncertainty constitutes the degree to which transient information asymmetries exist between potential exchange partners. 

I propose, second, that irreconcilable (i.e. non-transient) information asymmetries can persist in the market; however, above this threshold, variation in knowledge heterogeneity still occurs and has additional explanatory power with respect to the exchange behavior of market actors. Consider a common example from signaling theory. Consumers find it difficult (prohibitively costly) to discern the quality of a market offering ex ante--i.e. there exists some degree of information asymmetry in the potential exchange relationship. To overcome this problem, high quality producers invest in signals of quality. Perhaps they offer a warranty, and charge a slightly higher price. Low quality competitors are unable to do the same because it is more costly for them to offer a warranty than it is for the high-quality producer. Consumers value the signal as a way to discern the price-quality tradeoff and, in equilibrium, the high-quality producers are more profitable (Kirmani and Rao 2001; Spence 1973).

In the above example, all rational high-quality producers in a field will invest in signals of quality, and variations in knowledge heterogeneity (beyond that which is irreconcilable) will have minimal impact on exchange behavior. This is because reductions in information asymmetry lead to less investment in signaling, but not to changes in the set of partners chosen for exchange; less asymmetry reveals the same high-quality producers that were previously revealed via signals. However, this logic does not hold if acquisition of a signal is restricted by some criteria other than quality. If for instance, acquisition of a signal depends on position, legacy or capacity constraints, then it may not be available to all producers regardless of their quality.\footnote{See Podolny (2005) for a complete rendering of the status-based conception of Spence's (1973) signaling theory} 

In biotech, alliances with the National Institute of Health (NIH) can act as a signal of quality for fledgeling firms (Baum et al. 2000; Podolny 2001; Shane and Stuart 2002). However, the NIH has limited resources and cannot form alliances with all the high-quality biotech firms that exist. Thus, many equally (or more) qualified firms may enter the field, but lack the ability to signal quality through an alliance with the NIH. In cases where access to signals of quality are restricted, variations in knowledge heterogeneity will have a more profound impact on exchange. This is because the quality of market actors becomes more or less easy to evaluate independent of their signaling investment. 

\section{Context, Contributions and Chapter Structure}

My empirical setting is the biotechnology industry from 1991 to 2004. I chose this time period for two reasons. First, it represents a period in the industry with sparse analyst coverage and unreliable financial data--there were few individuals capable of evaluating the science, much less its commercial value (Powell et al. 2005; Wolfe 2001). These features of the industry amplified the importance of strategic alliances as signals of quality (Podolny 2001; Shane and Stuart 2002). Second, it predates widespread use of the internet as the primary medium for disseminating industry-specific news. This limits the variety of information sources shaping market actor's beliefs.

In the chapters that follow I investigate the biotechnology industry's tumultuous evolution and the stock market's reaction to it. To do so, I've assembled data from three distinct sources. I use the dataset collected by Powell and his colleagues (see Powell et al. 2005; Powell, Koput and Smith-Doerr 1996), extended through 2004 as the source for information on firm attributes and their alliances. Stock market data was collected from the Wharton Research Data Service (CRSP/Compustat merged) and supplemented with aggregate data from the New York Stock Exchange website. Textual data was collected from Bioworld, the predominant biotechnology trade journal during this time period, using the digital archives of Reuters, which now owns the journal. Measures of uncertainty are derived from textual analysis of the Bioworld archives and are shown to impact the trading volume, idiosyncratic risk and abnormal returns of publicly-traded biotechnology firms. These effects are moderated by a firm's position in a network of strategic alliances.

\subsection{Structural Signals}
In the classic model proposed by Spence (1973), a useful market signal must 1) be at least partially manipulable by the actor and, 2) the difficulty in obtaining the signal must be inversely correlated with the actor's quality level. Under this model, it is assumed that access to indicators of quality are equally available to all actors desiring to signal high quality. This assumption is valid in markets where the acquisition of signals is regulated or takes place through arms-length transactions. For instance, high-quality firms may spend more to advertise their products (Kirmani and Rao 2001). Presumably, the purchase of advertising space is based entirely on one's ability to pay for it. Indeed, the bidding and allocation of advertising space is often fully mediated by computers, thus removing many of the biases that might otherwise permeate human-to-human interactions. And yet, many markets {\it are} socially mediated and display biases based on gender (Koput and Gutek 2010), location (Owen-Smith and Powell 2004) and status (Kim and King 2014) to name just a few. 

An alternative model--inspired by Merton (1963) and articulated by Podolny (1993; 2005)--argues that the symbolic value of social position (i.e. structure) can affect exchange behavior independent of underlying quality. That is, given some baseline level of quality, benefits to those in a more favorable position will accrue at a faster rate than those in a less favorable position (Lynn, Podolny and Tao 2009). This rendering is mathematically consistent with the Spencian view if 1) the indicator of quality is not available to all actors desiring to signal high quality and, 2) the cost of production at a given level of quality is inversely associated with possession of the indicator. For instance, two equally qualified candidates may apply for a postdoc position at Stanford. The individual who receives the position benefits from a lower cost of producing research. This occurs because they have access to needed funding and are more likely to get invitations to present their research at other quality institutions (and thus receive useful feedback).

A steady stream of empirical research has advanced the social-structural view of market signals by documenting largely positive returns to advantageous structural positions (Benjamin and Podolny, 1999; Bothner, Kim, and Lee, 2014; Kim and King 2014; Podolny, 1993). However, recent theoretical arguments have criticized much of this work for its narrow focus on structural position as a signal of quality under uncertainty (Washington and Zajac, 2005; Jensen, Kim, and Kim, 2011). Accumulative advantage--or the Matthew Effect as it is commonly known--was originally postulated to affect returns based on a purely symbolic effect independent of quality, {\it over and above} any substantive effect that position has on subsequent quality. Thus, the purely symbolic portion of returns to a favorable position should exist even when underlying quality is fully known. Further, 

Malter (2014) confronts this challenge by studying whether a wine's status ranking affects the price at which it is sold. He uses the grand cru classification of the ch\^ateaux of the M\'edoc, which sorted 61 producers into five grands crus class\'es in 1855 as a basis for status and argues that underlying quality can be objectively distinguished from status via each wine's publicly available ratings (published online). However, he controls for the additional statistical bias of reverse causality by noting that the status ranking has not changed in 150 years (and doesn't change over the period of study). He thus makes an even stronger causal claim on the symbolic returns to structural advantage. (Malter 2014). 


However position can have signaling power without being theoretically confounded with reputation or quality if status orderings are based on legacy commitments.

Legacy escape the proxy for quality.

If status orderings are incorrect, then we'd expect an increase in consensus (reduction in uncertainty) to disproportionately go to younger firms.


Yet
recent work by Azoulay, Stuart, and Wang (2014) cast doubts on this . The authors match published articles written by scientists appointed as Howard Hughes Medical Investigators to those written by equally qualified candidates who were not appointed. They find that the symbolic effect of the appointment on citations is much smaller and more short-lived than expected given the strength of effects observed in previous studies. Their results call into question whether advantage can actually be accumulated based solely on symbolic tokens of status (cf. Merton 1968).


 Two other recent studies have attempted to address this criticism. Kim and King (2014) show that umpires who are uncertain about the quality of a baseball pitch tend to award more strikes to high-status pitchers. Because the quality of a pitch can be objectively determined ex post (via analysis of video), the authors can reliably distinguish between the effects of status and those due to underlying quality. Using similar arguments, 

two major criticisms. First, criticism for their unreliable statistical controls of quality (Simcoe and Waguespack, 2011). That is, measurements of favorable position are not sufficiently independent of quality such that position is merely a {\it reflection} of underlying quality rather than analytically distinct. When this is the case, assertions about the causal relationship between structural advantage and desirable outcomes are tenuous. 



However, by Azoulay, Stuart, and Wang (2014) cast doubts on this . The authors match published articles written by scientists appointed as Howard Hughes Medical Investigators to those written by equally qualified candidates who were not appointed. They find that the symbolic effect of the appointment on citations is much smaller and more short-lived than expected given the strength of effects observed in previous studies. Their results call into question whether advantage can actually be accumulated based solely on symbolic tokens of status (cf. Merton 1968).

Two other recent studies have attempted to address this criticism. Kim and King (2014) show that umpires who are uncertain about the quality of a baseball pitch tend to award more strikes to high-status pitchers. Because the quality of a pitch can be objectively determined ex post (via analysis of video), the authors can reliably distinguish between the effects of status and those due to underlying quality. Using similar arguments, 

Despite these findings, there remains significant variation in the magnitude and sustainability of structural advantage. If the findings of Azoulay, Stuart, and Wang (2014) can be generalized, the effect of a structural signal is short lived indeed. Yet, the findings of Kim and King (2014) and those of Malter (2014) seem to suggest that status rankings persist to such an extent that advantage can be disproportionately accumulated over time. These contradictions are not limited to studies at the individual level. Powell et al., (1999) show that firms, which occupy a central position in a network of strategic alliances accumulate advantage in both sales and growth. However, a followup study using similar data fails to document returns to centrality in the formation of new or repeat alliances (Powell et al., 2005). 

How can we reconcile these differences? Several recent theoretical arguments have criticized the structural view of market signals as too narrow because it fails to account for a purely symbolic effect of position (Washington and Zajac, 2005; Jensen, Kim, and Kim, 2011). Presumably, the symbolic effect is independent of, and unaffected by, any uncertainty about quality (Veblen 1899). Indeed, this is the argument made by Malter (2014) in his explanation how high-status wine producers are able to charge a price premium for their product. He 

 --presumably, a sy

 under uncertainty    the 

 Recall from my previous discussion that any benefit derived from possession of a structural indicator of quality is necessarily relies on the unobservability of quality. In fact, the value of such indicators should rise in the degree to which quality is unobservable (Podolny 2005).
 
primary contribution is to show that benefits vary not just with acquision of a signal but also with the observability of quality.

Despite this the laudable design of these studies, the structural view of signaling depends the unobservability of quality.

Goldberg varies the observability of quality but by exogenously determined classification
quality varies but not the observability of quality
Key to this 



\subsection{Outline}

Mathew effect from 2005 article not found

malter quality is unbiased because of online ratings... this is problematic because of audience.


Under the structural rendering of signaling theory, one thesis looms large. Benefits accrue to 

Despite this, there is still a great degree of controversy over of the causal relationship

Part of the problem has been an inability to 

A more complete rendering of this concept--including its operationalization--is presented in Chapter \ref{lang}.
I develop and test a set of hypotheses based on this reasoning in Chapter \ref{theory}.
-Practical and theoretical
-New leading indicator of market performance
-Methodological advances in the use of textual data in social science
-Empirical evidence of the feedback loops between behavior and belief.

