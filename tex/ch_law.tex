\chapter{Extending the Law of Exchange\label{law}}

\begin{small}
\begin{quote}
``It is a profoundly erroneous truism, repeated by all copy-books and by eminent people when they are making speeches, that I should cultivate the habit of thinking of what I are doing. The precise opposite is the case. Civilization advances by extending the number of important operations which I can perform without thinking about them."	
(Alfred North Whitehead 1911, p. 61)
\end{quote}
\end{small}

Marketing is operating in interesting and challenging times. Of cmyse one could argue this has always been the case; however, with the expansion of global competition, accelerating sustainability concerns and an increased focus on innovation, marketing and marketers are likely to confront an escalating set of challenges. Reibstein, Day and Wind (2009, p.1) sum it up Ill when they observe that marketing executives operate in an ``ambiguous, uncertain, fast-changing, and complex marketplace"--a statement echoed by a growing chorus of marketing scholars (see e.g. Achrol and Kotler 1999, 2012). my goal is to help make sense of these complex marketplaces through a deeper understanding of marketing as exchange. 

My contribution is conceptual in nature. I am providing readers with a novel way in which to ``envision" both markets and marketing (MacInnis 2011). To do so, I must delay the inclination to focus on matters of marketing management or public policy. If I rush too quickly towards normative theory, I are vulnerable to a fragmentation of thought more severe than Levitt (1960) discussed in his seminal article on marketing myopia. Indeed, I become \emph{market myopic} and fail to discover the generalities that apply to exchange in most (if not all) markets. I begin by reviewing the notion of exchange as the primary focus of marketing. 

\section{Exchange as the Focus of Marketing}
In 1965, Wroe Alderson laid the foundation for exchange as the focus of marketing with the ``law of exchange"  (Alderson 1965)--a contribution that stimulated substantial interest in a broader view of marketing (Kotler and Levy 1969; Kotler 1972).  Soon thereafter, a consensus emerged that the exchange concept was crucial to any complete understanding of marketing as a science (Bagozzi 1974, 1975; Houston and Gassenheimer 1987). This was a bold departure from 1930's thinking, when prominent organizations like the National Association of Marketing Teachers (NAMT) and the subsequent American Marketing Association (AMA) defined marketing as ``business activities involved in the flow of goods and services from production to consumption'' (AMA 1937; Lusch 2007). 

In 1985 the AMA adopted a revised definition of marketing that included the exchange of goods and services, but also ideas. Originally, the notion of idea exchange was focused on the marketing of social causes and the emerging development of social marketing (Kotler and Zaltman 1971). however, I consider ``ideas" and their ``distribution" in a less restrictive context. In fact I argue that the exchange of all market offerings through voluntary exchange requires the presence of shared ideas. Ideas--what I later refer to as an actor's set of representations--are more than just \emph{elements of exchange}. Rather it is the similarity/dissimilarity of these ideas (representations) in a population that facilitates the \emph{process of exchange} itself. 

Although the broadening of marketing had been discussed for nearly two decades (Kotler and Levy 1969) the revised AMA definition of marketing made the broadening of marketing official, and legitimized what Bagozzi (1974, p. 39) had observed a decade earlier,  ``...marketing is a general function of universal applicability. It is the discipline of exchange behavior, and it deals with problems related to this behavior." Bagozzi further speculated that the exchange concept could serve as the basis of ``the elusive `general theory of marketing.'" I do not propose a general theory of marketing here. However, I am able to show how an extended ``law of exchange" can be used to generate propositions that move us closer to a \emph{generalizable} understanding of markets and marketing. 

While largely skirting the issue of a general theory, there have been several notable attempts to embellish and amplify Alderson's law of exchange (Bagozzi 1978; Blacock and Wilkin 1979; Kotler 1984; Houston and Gassenheimer 1987). In one extension, Vargo and Lusch (2004a, 2008) argue that service for service exchange is a more fundamental process than exchange of physical goods--a line of reasoning, which follows (and extends) that of Bastiat (1848/1964). Service is defined as the application of resources (primarily knowledge and skills) for the benefit of another actor (Vargo and Lusch 2004a). Hence all actors reciprocally exchange specialized knowledge and skills; or more simply all actors do work for one another. When this exchange occurs through a tangible good, the good is an appliance that facilitates the distribution of a service (Vargo and Lusch 2004a, 2008). When this exchange occurs via a medium of exchange such as money, the money is simply a placeholder to subsequently exchange for service. Most importantly, this work lays the theoretical foundation for reclaiming the exchange concept from the producer-consumer distinction that has dominated mainstream theory for the last century (Vargo and Lusch 2011). Indeed, Alderson's (1965) law of exchange and the seminal work of Bagozzi (1974, 1975) makes no such distinction between one actor being a producer or seller, and the other being a consumer or buyer. 

Despite the promise of the exchange concept (and Alderson's proposed law), there are still major gaps in our understanding of the nature and process of exchange. For instance, there is a great deal of disagreement about the conditions under which exchange is more or less likely (Houston and Gassenheimer 1987). Because of this, it may be time to revisit Alderson's original proposition as part of a search for the deeper and broader understanding that was part of the exchange concept's initial promise. Assuming such an understanding is attainable, my claim is that it should be capable of transcending both time and context. For this to occur, my extension of Alderson's law of exchange will necessarily keep to a high level of abstraction. This begins with an abstract and abbreviated view of human exchange. 

\textbf{On Human Exchange}.
As humans began to specialize, their communities became more efficient; however, each individual also became more dependent on others in the process--those who specialized in one area, would rely on the specialized knowledge and skills of those in another area (Vargo and Lusch 2004a, 2008; Plato Book II of the Republic; Bastiat 1848/1964). This process led to an economy (society) of service for service exchange (Vargo and Lusch 2004a, 2008; Lusch and Vargo 2014). This in turn, requires the codification of ideas into a cognitive model I call a representation, and the subsequent development and diffusion of shared representations in a population (Sperber 1985). 

For the sake of clarity, I refer to an actor's collection of ideas (or mental models) as the set of representations that an actor has about the world. To the extent that any mental model persists for a relatively stable period of time, this definition of a representation is consistent with those found in a variety of social science traditions (see e.g. Bartlett 1932). In a more recent treatment, Loasby (1999) defines representations as those items within the brain, which facilitate action or decisions. I expand the Loasby definition slightly to include understanding as well. In other words, stimulus need not always lead to immediate action or decision; it may just be sensed, and stored in memory for use in future decisions (Denzau and North 1994). 

As Whitehead (1911) in the opening quote recognizes, the societal and individual benefits of specialization drive humans to continue developing means to economize on decision-making, information processing, and search (Loasby 1999). Humans accomplish this through the creation and diffusion of representations. Examples include ideologies, dominant logics, and paradigms (Denzau and North 1994; Sperber 1985). These representations (when shared by a community) coordinate the exchange of specialized knowledge and skills and their application for mutual advantage among human actors. 

As a practical matter, the distinction between types of representation is manifold; however, each type is also similar, but in a very general sense. Simply, all representations can be conceived of as relatively persistent collections of ideas maintained by human actors. Moreover, each actor's collection of ideas (representations) may be more or less similar to the collection of their neighbor's. I argue, that it is this rather basic principle--the similarity/dissimilarity of representations within a community of actors--that drives much of what we observe in the ambiguous, uncertain, fast-changing, and complex marketplace, and for that matter, in markets in general. My theory is thus one, which describes the homology of representation (i.e. the clustering of ideas or mental models) in a community of actors. This basic notion leads us logically towards positive predictions about the nature of markets, and towards normative prescriptions for marketing activity. The development of these ideas is presented in the subsequent sections.

\section{Theory Background}

How do market actors interface with their world or environment? Daft and Weick (1984) argue that managers are always unique in their interpretation of the environment. Day and Nedungadi (1994) demonstrate that managers form mental models (representations) of competitive advantage that influence strategic action. Fligstein (1996) theorizes that actors strive to ``create and maintain stable worlds within and across firms that allow firms to survive." This is a process that allows ``actors to interpret the actions of others and a reflection of how the market is structured" and to develop and enforce ``rules of exchange."  Rosa et al. (1999) argue that buyers and sellers jointly develop market representations that determine how the environment is subsequently viewed and enacted. Kjellberg and Helgesson (2007, p.143) suggest that ``markets are abstract entities and in order to speak of the market for a certain type of good, it is necessary to bridge temporal and spatial distances between individual exchanges and produce images of this market." Finally, Lusch and Vargo (2014) depict markets as dynamic constructions that are continuously being cocreated by market actors. 

If markets do not exist absent an experience and subsequent interpretation by market actors, then representation must be at the core of market constitution. All of the scholars referenced above, treat the environment as interpreted rather than fixed and absolute. While an objective reality (environment) may in fact exist, it is the actor's representation of this reality that facilitates its action and thus, behavior in a market context (North 1990). To further explore this line of thinking I expand upon the notion of the environment and how actors interface with an environment. 

I use the term ``environment" to indicate the context within which an actor functions or lives. I argue that interfacing with an environment requires sensing it. Indeed the notion of environment is only meaningful if there exists a sense-based, cognitive distinction between an actor and some stimulus perceived as different from the self (Luhman 2006). Further, given this distinction, the most basic (and incremental) departure from the act of sensing the environment, is that of sense making (Denzau and North 1994). In other words, humans form representations that ``make sense" of their surroundings. These are often stable, but also permeable, and subject to change over time (Bartlett 1932, Lynch and Schuler 1994, North 1990).
Simon (1978) suggests that the concept of rationality is ``economics" main export to other social sciences. The mind, however, is a scarce resource. How an actor rationalizes is thus of vital concern. By forming representations of the environment, actors reduce the effort required to judge suitability in the use of things--in consumption (Whitehead 1911). Representations are the ``how" of economic rationality. Indeed, representations provide the means with which actors rationalize consumption. This concept is illustrated in the following paragraphs using a simple, market-based example. 

My illustration could use any number of market offerings that have become pervasive or dominant in a community; however, here I use a ``chair." When I enter a restaurant, I know what to sit upon; in North America it is not a stool or the floor for instance. At the same time the restaurant manager or owner knew to have chairs available for restaurant customers. Once a representation is developed, the effort required to reproduce the concept is trivial. In my example, this efficiency is likely due to a sequence of chair-related sensing episodes (i.e. where learning has occurred), each of which adds to the robustness of the chair representation in memory. In time, one develops a quiver of such representations, which facilitate efficient interaction with the environment  and thus economize on the scarce resource of the mind. 

When a set of similar representations is held by a threshold number of actors in a community, that consistency of thought is the precursor to social coordination. Indeed, this homology in representation is stable because it has value beyond that which can be realized by any single actor. This is because it constitutes a coordination utility that benefits the community as a whole. This, I suggest, is the most basic form of network externality. For instance, one's concept of chair is made more useful by virtue of the fact that others share a similar representation of the object. When patrons enter a North American restaurant, little thought is wasted on how best to position oneself to eat. Moreover, North American restaurant owners waste little thought on whether to provide chairs for their patrons. As Humphreys (2010) has observed, prior to the introduction of an innovation (technology), consumers do not have a ``cognitive schema for understanding it" or the ``social norms for using it." 

As shared representations arise in a community, they become resources for human actors. As such, they can be drawn upon to economize on cognitive effort (as discussed above). But what sort of cognitive effort do shared representations minimize? Generally, it is the sort of thinking necessary to reduce uncertainty about future outcomes (Stinchcombe 1990). For instance, when I see the form of a chair, I have a certain degree of confidence (little uncertainty) that my full-weight will be supported should I choose to sit on it. One could argue that if an objective is known, then efficiencies that lead to uncertainty reduction in achieving the objective are the only type that matter (Loasby 1999). If a community of actors can standardize thinking (in the form of symbol, routine, custom or product), then they can accomplish this type of efficiency. 

Some representations can reach the stage where a community of actors shares them over a long period of time. These long-lived, shared representations can be thought of as institutions or even culture (Sperber 1985). Indeed, North (1991a) defines institutions as ``humanly devised constraints that structure political, economic and social interactions."  These may include norms, routines, laws or customs. however, each is developed and maintained as a means to coordinate human interaction so as to conserve the scarce resource of the mind (Loasby 1999). 

Based on the preceding reasoning one can argue that markets are more than the intersection of supply and demand (or a place where buyers and sellers come together).  Rather, markets are constituted by the set of shared representations in a community of actors. Or, as Humphreys (2010) observes, ``for a market to be created, producers and consumers must come to certain shared understandings of what is being exchanged and why." This market definition precedes one constituted by the intersection of supply and demand. Before supply and demand can unite, the representations of market actors need to unite. For example, the supply and demand for chairs (or what may be thought of as a chair market) depends crucially on some consensus regarding the representation of a chair. \footnote{Although I do not elaborate here, this consensus is also dependent on the use context of the object or in this case the ``chair." For instance, a representation of a suitable chair is different for a fine dining restaurant than for fast food. Nonetheless, shared representations are a necessary precondition for the emergence of a market.} I illustrate this point in the next section by demonstrating how even the most basic form of market exchange requires representational homology. 

\section{Extending the Law of Exchange}
I draw upon the prior theoretical discussion to extend Alderson's (1965, p. 84) ``law of exchange." It should be noted that Alderson deals with restricted exchange, which is dyadic or reciprocal, however, there are more complex types of exchange such as generalized exchange and complex exchange, both of which involve at least three actors (Bagozzi 1975). For the current discussion I focus on restricted exchange, as did Alderson.  In his law, the dyadic transfer of elements in an assortment may occur if the potency of each actor's assortment is increased. Alderson outlines the conditions for the occurrence of exchange between actors A and B as follows:

\begin{small}
...if X is an element in the assortment A, and Y is an element of the assortment B, then X is exchangeable for Y if, and only if, the following three conditions hold:
\begin{enumerate}
  \item X is different from Y
  \item The potency of the assortment A is increased by dropping X and adding Y, and
  \item The potency of the assortment B is increased by adding X and dropping Y.
\end{enumerate}
\end{small}

Alderson's term ``potency" is suitable in the abstract--and I will continue its use--though it is worth noting that subsequent work expanded on the original conception by referring to an actor's needs, desires, utility, value and benefits (Bagozzi 1975; Houston and Gassenheimer 1987; Kotler 1984). Kotler (1984) simply states that each actor must have something that might be of value to the other. Regardless, potency is always treated as something that is uniquely assessed by each actor. And yet, each expects to be better off after an exchange. This is because the item that each actor received is valued more than what they gave up. Mutual gain thus becomes the glue that holds voluntary exchange systems together. 

Alderson and Miles (1965) further develop Alderson's law of exchange. First, they incorporate the cost of executing an exchange transaction. Second, they allow for each actor to have outside options such that exchange occurs only if both of the actors prefer it to any other alternative. Third, they address the issue of exchange that occurs through a supply chain or distribution channel. 

Other scholars help us further understand the law of exchange by including additional necessary conditions such as the notion that ``...humans are able to anticipate the consequences of their actions", and ``direct their behaviors toward their preferred anticipated consequences" and ``create innovative behaviors that are aimed toward the consequences they desire" (Blalock and Wilken 1979, pp. 29-30).  Further, Kotler (1984) states additional conditions which include: ``each party is capable of communication and delivery" and ``each party is free to accept or reject the offer" and ``each party believes it is appropriate or desirable to deal with the other party" (Kotler 1984). However, despite these additions, each subsequent treatment of dyadic exchange assumes that actors understand a priori: 
\begin{enumerate}
  \item The manner in which X differs from Y.
  \item How X or Y can increase the potency of an assortment.
  \item That actor A is not only capable of delivery of X, and B delivery of Y, but that they will deliver. 
\end{enumerate}

Upon which grounds are these knowledge claims based? It may seem a trivial assertion, but the argument as it stands evokes consideration of the neoclassical economic man. Indeed, the troublesome assumption here is that of perfect information--a situation in which actors understand the nature of their own wants and needs, how the element offered by a potential exchange partner can address them, and whether the element will be delivered as promised. Indeed, perfect information is the logical opposite of complete uncertainty. The requirements enumerated above suggest the need for just the sort of uncertainty reducing function that shared representations provide. 

The conditions for exchange posed by Alderson (1965) and those expanding upon his arguments (e.g. Blalock and Wilken 1979; Kotler 1984; Houston and Gassenheimer 1987), are necessary, but insufficient prerequisites to voluntary dyadic exchange. Kotler (1984) recognizes that each party must (at a minimum) be capable of communicating their intentions. This partially addresses problems related to ``the manner in which X differs from Y'' and ``how X or Y can increase the potency of an assortment." However, as I will discuss below, the shared representations of actors in an exchange--and in the relevant community of actors--is more primary and a precursor to the sort of communication described by Kotler (1984). Kotler (1984) also argues each actor must be capable of delivery, however, enforcement of such behavior is determined by shared social norms (Granovotter 1985) as defined by for instance, religion, polity or culture among a community of actors (Lusch and Vargo 2014). 

\textbf{How is X different from Y?}
\begin{small}
\begin{quote}
``There is not a sentence which adequately states its own meaning. There is always a background of presupposition which defies analysis by reason of its infinitude." (Whitehead 1951, p. 699)
\end{quote}
\end{small}

The conditions for exchange outlined above imply that actors are heterogeneous in the resources they possess and in how they uniquely value elements of the exchange. That X is different from Y (in the Alderson framework) is an artifact of human's ability to notice a difference in the first place. Loasby (1999, p. 32) states, ``...what do economists fail to economise? The answer is the scarce resource of cognition." Our limited cognitive abilities prevent us from processing and storing the vast quantities of information present in our environment (Loasby 1999; March and Simon 1958). In response, individuals construct representations as a way to make sense of the world (Denzau and North 1994). Due to human's limited cognitive ability, these representations necessarily summarize experience (Arthur 2009; Loasby 1991; Loasby 1999; Knight 1921; Whitehead 1911; Whitehead 1951)--that is, they are imperfect and unique. Exactly how X is different from Y depends entirely on the representations each actor maintains. For instance, a toddler can nary tell the difference between a cell phone and a bar of soap. For her, the representations of each object have yet to diverge from the more primitive representation of something rectangular in shape and of a certain weight. 

Given the personal and familiar circumstances surrounding most exchange, it is easy to see how Alderson's logic (that elements of exchange have to be different), can be taken for granted--it is quite natural to think in terms of the difference between for instance, a dollar bill and a chocolate bar. however, recent advances in marketing theory have excoriated these distinctions by exploring the way in which ``indirect exchange masks the fundamental basis of exchange" (Vargo and Lusch 2008). The elements of trade are superficial--they are merely the medium through which an offering is presented. Giddens (1991, 22) discusses symbolic tokens, or the ``media of interchange which can be `passed around' without regard to the specific characteristics of individuals or groups that handle them at any particular juncture." A symbolic token dis-embedded from reality doesn't need a reality to be understood (Giddens 1991, 22; Loebler and Lusch 2014). Symbolic tokens are similar to mediated currencies. ``They refer to something but they do not need the referred object to be exchanged. Money is an example of a symbolic token; it is commonly understood what \$5 means" (Lobler and Lusch 2014). As such, money is a placeholder to be further exchanged for objects that increase assortment potency. 

Given our bias as participants in the marketplace, it may be useful to adopt a toy (mental) model that defies common everyday experience but may still be somewhat familiar. Children often like to gather and exchange rocks. For this purpose, let us imagine two child actors (A1 and A2), who each maintain an assortment of rocks. One can then rightly ask, under what conditions will A1 exchange one of her rocks for one or more from A2's assortment? Given my previous discussion, a necessary precondition is that A1 forms representations of the rocks in her assortment. She could for instance experience that some rocks are larger than others. This experience leads to the formulation of a ``size" representation with certain causal relations like, if I encounter a big rock it is heavier. This representation can then be deployed to categorize and impart meaning on future encounters with rocks, such as those in A2's assortment. If A2 has developed the same (or similar) representation from his experience, then the condition that X is different from Y is clearly met from the perspective of both actors eyeing each other's assortment.

Alternatively, each actor may develop orthogonal representations of their respective assortment. Whereas A1 develops a representation of ``size," A2 develops a representation of ``shape." In eyeing each other's assortment, the condition that X is different from Y is still met from the perspective of each actor. However, here the condition that ``each party is capable of communication and delivery" (i.e. Kotler 1984 p. 8) becomes problematic. Without shared representation it is impossible to coordinate (Shannon 1948). In other words, human representation is the facility with which meaning is applied to a message. With orthogonal representations, there is no basis for coordination. Since I am speaking of voluntary dyadic exchange (what Bagozzi 1975 refers to as restricted exchange), then coordination is a necessary condition as Kotler (1984) rightly point out. however, the conditions for coordination are rooted in representational homology, a subject, which has yet to be explored in the context of exchange.

\textbf{How can X or Y increase the potency of my assortment?}
Let us continue my example by considering a primitive form of coordination and the shared representations that it requires. Given my two actors and the aforementioned set of orthogonal representations, one could argue that exchange is still possible if an actor were to physically point at the elements of potential trade. A1 need not know of A2's ``shape" representation to decipher his intent. However, in this modification, I have unintentionally forced another form of representation into my toy mental model--simply, the representation of ``pointing" held by each actor. It is entirely possible and reasonable to assume that actors also have orthogonal representations of pointing behavior, which in turn prevents coordination in the same way as orthogonal representations of element features. To one actor, pointing may represent a sign that indicates this is my most valuable rock and it is not available for exchange. To the other actor, pointing may signal that this is the rock I would like to most rid myself of. Therefore, we never free ourselves from the constraint of shared representation as a precondition for exchange. 

\textbf{Will they deliver?}
Humans however, are fairly clever at inventing ways to coordinate. As I discussed above, shared representations arise as mechanisms to promote ordered behavior. While representations as cognitive structures are internal to the actor, institutions, by definition subsume the representations of an individual (Denzau and North 1994). Thus the act of pointing may become an institution and its meaning tacitly agreed on. In this example, an institution arises as a widespread, long-lived, shared representation, embedded in a community of actors. In the current example, think of pointing perhaps as analogous to a contractual warranty or title. These are essentially institutionalized solutions to handle uncertainty in the delivery of a value proposition. 

While it is not necessary that all representations are shared by both actors in dyadic exchange, it is necessary that some critical proportion are shared. Thus for exchange to occur at all, there needs to be some overlap in representation between two actors (for restricted or dyadic exchange) or some concentration of representation in a community (for exchange in general). Moreover, I argue that the distinction between the elements of exchange (e.g. rocks) and the institutions (e.g. pointing) is non-essential since I can conveniently subsume both under my definition of representation. 

The important intuition is that I recognize the role of shared representations in coordination--one need not understand all features of an element as a prerequisite to making a purchase (economic exchange). It is often the case that just the price tag is sufficient. The behavior of an actor in my example pointing at two rocks is comparable to that which is conveyed by a price tag. The pointing behavior shows one actor's notion of rock equivalence in the same way that a price tag connotes a retailer's notion of equivalence between an object and currency. The pointing process must be interpreted in the same way that a price tag must be interpreted, and each is dependent on the element to which it refers for meaning. This dependency renders the representation of ``pointing" inseparable from the representation of rock in the same way that a representation of price is inseparable from the object to which it refers (Erickson and Johansson 1985). 

The potential increases in potency that motivate an exchange can be conceptualized in a similar manner. That a large rock is suitable as a paperIight is part of its representation and connotes meaning in the same way that the action of pointing or a price tag on an object does. Whether an element has the capacity to increase potency is fully dependent on the representations of the actor in possession of the assortment. However, an understanding of how a recipient intends to value an element can facilitate the exchange of one for another by providing a mode of coordination via the notion of equivalence. Paradoxically, it is the similarity of thought (representation) between two actors that allows for any communication of the manner in which X differs from Y, and thus the way in which a new element might increase potency. The representations shared by two actors form the basis for a definition of how one element differs from another (Humphreys 2010). Without a perceptible difference, there can be no conception of value or changes in potency. 

\section{A Shared Representations Model of Exchange}
\begin{small}
\begin{quote}
 ``I would argue...that the concept `redundancy' is at least a partial synonym of `meaning.' As I see it, if the receiver can guess at missing parts of the message, then those parts which are received must, in fact, carry a meaning which refers to the missing parts and is information about those parts." (Bateson 1972, p. 420)
 \end{quote}
 \end{small}
 
\textbf{Market Entropy}
In my toy model above, actors are in possession of only a small number of representations. For obvious reasons, this is an unrealistic model. Rather, it is likely that actors hold many representations and that some of these representations are the same or similar to those held by other actors. In a more realistic vision of the world, I need some way of describing the similarity of representation in a community of actors. This formalization step is necessary to derive statements about the effect of similarity on various outcomes of interest. For the purposes of this explanatory effort, I can model representational homology in terms of the entropy characterizing an exchange system ($I_E$) in which each actor maintains one of a variety of representations about some abstract element of exchange.  
 
My notion of market entropy is defined formally below; however, intuitively it can be viewed as the amount of representational disorder in a community of actors. While technically consistent, the term ``disorder" is a bit of a misnomer in this case. As I use it, ``disorder" is actually equivalent to \emph{increases} in similarity of representation. To illustrate, I turn to an example of entropy used in physics. Imagine a large, clear glass jar filled with water. Then imagine that you insert one drop of red food coloring on one side of the jar and one drop of blue food coloring on the opposite side of the jar. At first, the colors from the red and blue droplets are relatively confined to their respective points of entry. However, given enough time, each droplet diffuses throughout the glass such that all of the water appears purple to an outside observer. 
 
In the above example, the system (of water and food coloring) moved from a state of order (low entropy) to one of disorder (high entropy). That is, the colors were separate and ordered to begin with, but ended up mixed and disordered by the end. If I imagine my community of actors to be the water in this imaginary glass, and each colored droplet to be a representation, then it is easy to see how representational homology is actually the result of mixing, and hence, disorder. In the beginning, actors maintain separate (red and blue) representations, but over time these representations are mixed together and shared.
 
I formalize this analogy by relating it to information theory in the following manner. At the beginning of my example, randomly choosing a few water molecules and observing their color gave us reliable information about which side of the glass they were chosen from. However, by the end of my example, observing the color of a few randomly chosen water molecules gave us no information about which side of the glass they were selected from.  I can consider my market $\Omega$, as some relevant population of human actors where $\{A\} = \{A_1, A_2,\ldots,A_n\}$ is an n-cell partition of $\Omega$ and can be interpreted as a grouping of individual actors into meaningful segments. $\{R\} = \{R_1, R_2,\ldots,R_m\}$ is an m-cell partition of $\Omega$, and can be interpreted as a set of distinct representations about some abstract element of exchange.  The information available in this kind of system can be described by,

\begin{equation}
I_E = H(A)+H(R)-H(A \times R)
\end{equation}

$H(A)$ is the uncertainty in the segmentation variable (Shannon 1948; Ramaswamy et. al, 1993), 

\begin{equation}
\nonumber H(A) = \sum_{i=1}^n -p_i ln(p_i),
\end{equation}

\noindent where, $p_i$ is the unconditional probability of choosing an $A_i$-type actor at random from the domain population. Similarly, $H(R)$ is the uncertainty in the set of representation portfolios held by the community,

\begin{equation}
\nonumber H(R) = \sum_{j=1}^m -p_j ln(p_j),
\end{equation}

\noindent where, $p_j$ is the unconditional probability that a randomly selected actor has portfolio $R_j$. $H(A \times R)$ is the uncertainty of the system,

\begin{equation}
\nonumber H(A \times R) = \sum_{i=1}^n\sum_{j=1}^m -p_ip_{ij}ln(p_ip_{ij}),
\end{equation}

\noindent where $p_{ij}$ is the conditional probability of $R_j$ given $A_i$ (Greeno 1970). It can be shown that $H(A \times R)$ is equal to $H(A) + H(R)$ when the partition variables are independent (i.e. $p_{ij} = p_j$ for all $i$). This means that the information in the system $I_E = 0$ and market entropy is at it's maximum. For a population of two, this is equivalent to saying that both actors share the exact same representations (e.g. there is full overlap or full homology). In contrast, the case in which all conditional probabilities are either 1 or 0 implies no overlap in representations (no homology) between two actors. In this case $I_E$ takes on its maximal value and market entropy is minimized.
 
To illustrate, consider a domain with two actors segmented into types $\{A_1, A_2\}$ and two representations $\{R_1, R_2\}$.  One possible empirical reality is,

\begin{center}
\begin{tabular}{ l l l l}
& & $R_1$ & $R_2$ \\
$P(A_i \cap R_j)=$ & $A_1$ & $0.50$ & $0.00$ \\
 & $A_2$ & $0.00$ & $0.50$
\end{tabular}
\end{center}

\noindent Here, 

\begin{equation}
\nonumber H(A) = H(R) = 2(-0.50)ln(0.50) = 0.69,
\end{equation}

\noindent but,

\begin{equation}
\nonumber H(A \times R) = 2[(-0.50)ln(0.50) + (0.00)ln(0.00)].
\end{equation}

Thus, $H(A \times R)$ reduces to $H(A)$ and the value of $I_E$ goes to it's maximum value,

\begin{equation}
\nonumber 0.69 + 0.69 - 0.69 = 0.69.
\end{equation}

In other words, the partition of $\Omega$ by $\{A\}$ gives us the maximum information possible about the differences between each actor with respect to $R$. An alternative empirical reality is, 

\begin{center}
\begin{tabular}{ l l l l}
& & $R_1$ & $R_2$ \\
$P(A_i \cap R_j)=$ & $A_1$ & $0.50$ & $0.00$ \\
 & $A_2$ & $0.50$ & $0.00$
\end{tabular}
\end{center}

In this scenario, $p_ij$ clearly equals $p_j$ for all $i$, such that $H(A \times R) = H(A) + H(R)$ and $I_E = 0$. In other words, actors drawn at random from this community maintain $R_1$ and $R_2$ with equal likelihood regardless of which segment they are from. $\{A\}$ provide no insight about $\{R\}$. This result follows from the earlier claim that there will be full overlap in a two-actor framework when my segmentation variables are independent. Clearly, both actors have $R_1$ but not $R_2$.
 
The results expressed in the preceding sections follow from intuition--a maximal entropy value describes situations where actors share completely their representations (the highest level of representational homology). Consider again my example of the chair market in North America. If I pick two actors at random, the likelihood that they share a similar representation of chairs is quite high. Information about chairs in this market is diffuse, unordered and high entropy. In contrast, a low level of market entropy ($max(I_E)$) describes minimal sharing (low representational homology). If I expand my chair market to include Japan, the likelihood that two actors picked at random from this new population also share a similar representation of chairs is somewhat lower. Information in this new market is more structured, less diffuse and thus lower in market entropy. 
 
Low entropy markets hinder exchange because market actors find it difficult to coordinate. For exchange to occur more readily, some population of actors must adopt (or adapt to) the representations of other market actors. Of course the process through which this occurs can be bi-directional--actors may incorporate parts of one representation and discard others (e.g. the water in the jar turns purple rather than red or blue); however, the important intuition is that under low entropy, some process that encourages similarity (the mixing of ideas) is required in order to drive further exchange.
 
\begin{prop}
An increase in representational homology (similarity) drives exchange when market entropy is low.
\end{prop}

Too much representational homology however, may hinder exchange. To illustrate, let us revisit my two-actor model and assume max entropy. This means that representations of value, equivalence and element properties (and all else for that matter) are identical for both actors. What differs, is only the composition of each actor's cache of potential resources. For instance, one actor may have all large rocks and another may have all small rocks. Since both actors share exactly, any representations that contribute to potential potency enhancements, all I need to do is identify one example of potency to understand how the process plays out. For instance, assume that potency is determined by maximizing variance across size. Since each actor has rocks of different sizes, I would expect exchanges to occur until approximately half of each actor's assortment is transferred to the other's assortment. However, at this point in my example, the two-actor system has reached equilibrium and further exchange will not occur (see e.g. Milgrom and Stokey 1982 for how this phenomenon manifests in financial equity markets). In order to prompt further exchange, one needs to impose a different set of representations. For instance, if I exogenously define potency in terms of color variance instead of size variance, my two-actor system will again begin to exchange.
 
\begin{prop}
A decrease in representational homology (novelty) drives exchange when market entropy is high.
\end{prop}

\textbf{Alpha and Beta Markets}.
Based on the preceding arguments and discussion, I can make the claim that all markets have two ``extreme," equilibrium conditions. One equilibrium condition occurs under minimum market entropy; I will label this as an alpha market. The other equilibrium condition occurs when entropy is maximized; I will label this as a beta market. Of course the extreme forms of alpha and beta markets don't exist. however, one can view a market as evolving along a continuum between these extremes--that is, moving on a path toward a min-entropy (alpha) market with high differentiation, toward a max-entropy (beta) market with little differentiation, or fluctuating between the two. When engaging with an alpha market, the major function of marketing is to foster more shared experiences (i.e. similarity). On the other hand, when engaging with a beta market, the major function of marketing is to introduce new representations (i.e. novelty). 
 
Novelty however, takes time--it needs to diffuse. Traditionally marketers have characterized this behavior in terms of a product life cycle (Gardner 1987). however, I argue that it more accurately be vieId as the epidemiology of a representation (Sperber 1985). Recall the droplets of food coloring from my previous example of market entropy. It is not the physical manifestation of a product that is spreading, but rather the mental model (representation) of that product--the market offering. 

A recent example is the introduction of the iPhone. This novel representation changed what had become a high-entropy, undifferentiated (beta) cell phone market, into a low entropy, differentiated (alpha) market. By simply observing a selection of cell phones, one could clearly distinguish which came from Apple. Yet, once this new representation was introduced, market actors (both buyers and sellers) rapidly adopted the novel representation to such an extent that it became the dominant cell phone design. Several firms sought to make their phones ``similar" in appearance and function to an iPhone. The result is what we now call the ``smart phone market," which itself is now returning to a high-entropy, undifferentiated (beta) market. That is, cell phones have again become similar in both appearance and functionality.


\section{Discussion}
In this article I have argued that exchange is the product of representational homology and that such an argument provides a unified approach to the study of markets. The exchange concept is so embedded in marketing scholarship and practice that I often fail to address its properties independent of the sub discipline in which it is addressed. Because of this, marketing practitioners operate in theoretical silos defined by area of focus, whether new product development, communications or strategy. In the sections that follow, I discuss the ways in which my proposed model of exchange can be used to understand the strategic decisions employed across a variety of marketing practices.
Broad Implications

The proposed shared representations model of exchange, an extension of Alderson?s law of exchange, has multiple implications for the practice of marketing. In that regard, one implication appears foremost?an understanding of market entropy should precede the formulation of marketing strategy. This is because the characteristics of alpha and beta markets (and the extent to which a market is trending one direction or the other) can inform the type of strategy that should be pursued. In brief, when market entropy is high (undifferentiated), strategy should orient towards innovation and novelty. In contrast, when market entropy is low (highly differentiated), strategy should orient towards diffusion and similarity.  

The practice of marketing is further affected by the focus on shared representations (representational homology) as a foundational concept of interest. It moves the marketing discipline away from a product centric focus, and (to some degree) away from a customer focus. In their place is a focus on generic market actors and generic market offerings. To illustrate this type of thinking I can ask for instance: ?how does the marketplace represent an automobile, cell phone, or college?? The focus, at least initially, is not on the brand?as in Mercedes, iPhone or Harvard?but rather on the way in which market actors represent the fundamental concepts of transportation, communication and education. Do their representations overlap? Where are they different? What opportunities exist in the ansIrs to these questions?     

Innovation. 
When should a firm attempt incremental vs. radical or disruptive innovation? From an abstract and propositional knowledge perspective, this question was previously ansIred. Proposition 2 suggests that increasing representational homology (similarity) will drive exchange when market entropy is low. Thus, in highly differentiated (alpha) markets, exchange will be driven by incremental innovation. On the other hand, Proposition 3 recommends the introduction of new representations (i.e. novelty) or what can be thought of as radical design when market entropy is high. Thus, in undifferentiated (beta) markets, exchange will be driven by more aggressive innovation. 

New product and service innovations often exist at the confluence of two previously unrelated industries (Arthur 2009). Consider for instance, the recent emergence of the so-called ?sharing economy.? Services like Uber  and AirBNB  have succeeded largely by merging developments in information technology with traditional offerings in transportation and hospitality. Yet, the ?disruption? of these beta markets necessarily required a new representation of the market for cabs and hotels. The resultant increase in exchange stems from an initial decrease in market entropy folloId by the gradual acceptance by market participants of the process through which these services operate. NeIr services like tool-sharing service Streetbank  have subsequently leveraged the market representation of a ?sharing economy? to expand their business as entropy increases and the market again equilibrates to beta. 

Marketing Communication. 
In practice, material innovation considered ?radical,? can be quite difficult to achieve. This is especially evident in markets defined by commodity resources like gasoline or aspirin.  When confronting these undifferentiated (beta) markets, communication strategies that emphasize the uniqueness of market offerings can help drive exchange. Since the material market offerings are undifferentiated, marketers should foster uniqueness by focusing on intangibles such as brand image. Intangibles are often more difficult for competitors to copy (Teece 1986) and can thus better sustain exchange in a beta market. For instance consider aspirin, baby oil, blue denim jeans, gasoline, or many kitchen appliances. The dissimilarity of offerings in each of these markets is the result of considerable investments in brand image. Consequently the brands of Bayer, Johnson \& Johnson, Levi, Shell, and Kitchen Aid constitute a smyce of differentiation in an otherwise beta market. 

In contrast, marketers dealing with highly differentiated (alpha) markets, should use communication strategies that emphasize the similarity of market offerings. Granted, some differentiation can remain (e.g. a loIr price); however, a competitive market offering must be substitutable by definition. Recall from the discussion preceding Proposition 1, that recognition of element differences are (paradoxically) based initially on some notion of equivalence. If a competitor offers a loIr price, the rest of their market offering must be similar enough to the offering with a higher price, that the difference can be meaningfully evaluated. In an alpha market, this baseline for comparison across market offerings cannot be taken for granted.

Beyond Supply Chains and Channels. 
Marketing management develops strategy and tactics for the flow of goods and services between market actors. Traditionally, this ?flow? has been represented as a series of dyads that result in a chain or channel. however, increasingly scholars are applying concepts like value constellations (Normann and Ramirez 1993), actor-to-actor networks (Vargo and Lusch 2011), and ecosystems (Mars et al. 2012) to solve the problems of supply and distribution. This broadened perspective fits Ill with the representational homology view of markets. Shared representations of the marketplace constitute the glue, which holds these constellations, networks, and ecosystems together. As Vargo and Lusch (2011) suggest, shared institutional logics provide the norms, rules and customs that facilitate trust and thus allow actors to efficiently cooperate in the exchange of service. however the level of efficiency realized depends on the level of representational homology present in the exchange context. Trust can be defined as the certainty one actor has about the behavior of another. As argued previously, this certainty is dependent on a shared understanding of the rules that govern exchange behavior. 

\bf{Research Requirements/Tools }
Virtually all of the preceding implications for marketing practice require new research and measurement techniques/tools. Foremost, I need to be able to ansIr the following, ?how does one go about measuring the concept of representational homology and hence market entropy?? I offer some suggestions and also recommend further work that needs to be done to more fully ansIr this important question. 

Entropy Measurement. A preliminary but vital question is if entropy is an appropriate measure to capture representational homology. I believe it is a valid measure because it captures variation independent of scale and it is grounded in theory. In this regard I are not relying on the concept of entropy from thermodynamics, which posits that in a closed system energy will be used up as entropy inevitably increases. Notably, some economists (see e.g., Georgescu-Roegen 1971; Rifkin 1980) and macro-marketers (Kilbmyne et al. 1997) are focused on this concept of entropy. my focus is instead on the notion of entropy as developed in information and communications theory. This is appropriate because it allows for new information in the form of surprise or novelty and thus the reduction of entropy (the creation of order) as a counterpoint to the absence of information (max entropy). Since market representations are not static, this conceptualization also allows us to capture the non-equilibrium nature of markets in practice?the notion that markets generally move between order and disorder in response to information shocks (Hunt 1999; Dickson 1992). 

The entropy concept, as I use it, is not new to marketing and has already provided valuable insights. For instance Ramaswamy et al. (1993) use an entropy measure to determine if market segments are reliably distinct. They find that when entropy is at its maximum it is not possible to make reliable distinctions between segments. Additional research using an entropy measure in market segmentation studies has produced similar results (DeSarbo et al. 1995; Kamakura and Idel 1995; Hofstede et al. 1999). Further research suggests entropy as the primary diagnostic statistic to ensure separation of parameters defining segments (Idel and Kamakura 2000). my model complements that of Ramaswamy et al. (1993) and subsequent researchers by noting that in high entropy, undifferentiated (beta) markets, the representations of market actors are highly homogeneous. I depart from prior research in my use of the entropy concept to motivate theory on exchange.
Fine vs. Coarse Grain Measurement. It is unrealistic and unnecessary to examine every minor nuance of a representation. Thus, I suggest identifying the key elements (or features) of a shared representation. This process of ?coarse graining,??while often involving substantial intuition on the part of the researcher?is one which social scientists are keenly familiar (Stewart 1981). A key element is one that if it Ire not included in the final formulation, the representation would have substantially less meaning and, thus rendered unrecognizable by market actors. Consequently, a priority in developing measurement models of shared representations is to identify how to meaningfully isolate the key elements. 

At the most fine-grained level, elements that comprise a representation are very numerous, and yet, may not be noticeable by humans. Consider for instance, the representation of red wine. With the exception of professional wine tasters, most individuals are unable to tell the difference between an $8 bottle and a $25 bottle. In product testing laboratories one can use electron microscopy to view defects?for example, in leather used to cover a car seat. While these defects may not be visible to most humans, it is possible that such microscopic elements collectively create an overall impression of quality?especially when combined with other design elements. For instance in the leather car seat example, if one combines a representation of the car flooring, side panels and dash, then a more ?coarse-grained? representation may be formed. Understanding the separation between fine and coarse-grained representations is crucial for undertaking research in this area.
Beyond Envisioning 

	From the outset I emphasized that my conceptual framework is intended to help marketing scholars and practitioners envision markets and marketing in a way that is both different and more useful. To accomplish this I took an abstract approach that can be used as a platform for more general theories of marketing. I hope this approach has enabled one to envision markets and marketing as a function of representational homology in a community of actors. The next step is to ratchet down from this high level of abstraction; however, this requires middle-range theory. Such middle-range theories are often industry specific. For instance, a theory focused on financial services industry might include explanations of how actors socially construct markets, how firms practice marketing, and the representations that comprise the macro-marketing system. Alternatively, a middle-range theory can be developed around a particular marketing function like inventory management or promotion. Middle-range theory can also focus on different contexts like those that arise in the domain of international marketing. 

	Given that the extended law of exchange is abstract, further theorizing and testing is not constrained by a particular logic of discovery or logic of justification. Indeed multiple theoretical frameworks including historical, cultural, psychological, economic and others are Ilcome. Moreover, suitable empirical evidence can come from ethnographies, historical analysis, experiments, surveys, archival records or big data. 

\subsection{Concluding Comments}
While many in the field have called for more applied, transformative, and managerial research, I suggest that of at least equal importance is the need for more foundational theory. In marketing I argue that this foundational theory should be built around exchange. The conceptual framework offered herein, can help scholars and practitioners envision markets and marketing as grounded in shared representations between market actors. 
	
I argue that with this type of foundational knowledge of exchange, the discipline can actually be more applied. This is because good theory can provide marketing practitioners and public policy makers with something to apply. I recognize that there are shortcomings to the extended law of exchange, but also see these shortcomings as a call for more research on exchange and exchange systems. Much of this research needs to be at the level of mid-range theory that can be developed based on the representational model of exchange.
