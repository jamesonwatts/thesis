\chapter{Measuring Uncertainty Using Language\label{lang}}

\section{Introduction}
Chapter \ref{law} introduced an information theoretic definition of knowledge heterogeneity based on the concept of entropy. As noted in section 2.4, it can be tempting to conflate high levels of market entropy with the concept of uncertainty. This stems from the mathematical definition of high entropy as a state with little information (i.e. high disorder). However, information (under the proposed definition) refers to the \emph{differences} between market actors. When we can distinguish one actor from the next, we have some information about the market. When we can't, we have no information. Thus, it is meaningful differences (not similarity) that are the source of mathematical order in the market entropy construct. 

Nonetheless, knowledge heterogeneity may describe a particular type of environmental uncertainty. For instance, the uncertainty that results from disagreement over the importance (or appropriateness) of market behaviors (practices, products, etc.). As alluded to in the introduction, this will naturally occur during transitions of technology, taste and attention if innovations are assumed to arise locally and diffuse gradually (Colyvas 2007; Rogers 2002). For a period following the innovation, the variance in knowledge and opinion of that innovation will be more pronounced.

In the sections that follow I review the literature on environmental uncertainty and its measurement. Drawing on social constructionist theories and insights from research on the legitimation process (Hannan and Carroll 1992; Rosa et al., 1999), I then propose a new measure based on changes in language use over time. Finally, I show how this measure is related to exchange behavior in the biotechnology industry.

\section{Environmental Uncertainty, Legitimacy and Language}
\begin{small}
\begin{quote}
``Grammars in both cooking and engineering exist not just as rules but as a set of unspoken practices taken for granted" (Arthur 2009 p. 77)
\end{quote}
\end{small}

Capturing the effects of environmental uncertainty over time is no easy task. Indeed, much of the noteworthy empirical work in this area defines uncertainty according to some exogenously determined classification, and often one that is fixed in time. For instance, Podolny (1994) examines exchange relationships in the issuance of investment-grade vs. non-investment-grade debt. Firms operating in the latter context are assumed to experience higher levels of uncertainty. In a more contemporary rendering, Erdem et al. (2006) use a country's predetermined score on an uncertainty avoidance scale to make claims about how individuals in that country react to signals of brand credibility. 

Studies that attempt to track environmental uncertainty over time, often do so using measures of volatility in a portfolio of stocks (see e.g. Beckman et al. 2004). However, this might be better understood as systematic risk (Thomaz and Swaminathan 2014). While it may be a consequence of uncertainty, risk is theoretically distinct; under risk, the probabilities associated with future outcomes are assumed to be known, whereas uncertain prospects have unknown probabilities (Tversky and Fox 1995). More importantly, the stock market represents a limited sample of market participants and likely misses important behavior undertaken by smaller or newer participants in a field. 

Other studies have looked at the percentage change in an industry's patenting activity as a way to capture technical uncertainty (see e.g. Goerzen 2007; Luque 2002). However, technical uncertainty is only one dimension of environmental uncertainty and patents are only one facet of market behavior. Moreover, patenting activity is itself subject to changes in practice over time. For example, biotech firms have learned to over-apply for patents as a way to conceal the innovations they are actually using (Wolfe 2001).

A parallel stream of research has examined the related concept of legitimacy, which can be defined as ``consensus among agents (audiences) that the features and activities of [market actors] are appropriate and desirable within a widespread, taken-for-granted system of norms or social codes" (Cattani et al. 2008, p. 147). Legitimacy is linked to uncertainty through the legitimation process. Innovations in behavior arise locally and spread gradually to others in a field (Colyvas 2007). As this process unfolds, there are periods of high disagreement about which behaviors are appropriate (or optimal). This in turn, increases the uncertainty actors have about market reactions to future behavior.

This formulation of environmental uncertainty--derived from consideration of the legitimation process--comports nicely with my focus on field-level transitions in technology, taste and attention. However, a brief review of the empirical literature on legitimacy raises an additional concern: the behavior deemed legitimate is typically defined ex post--that is, operationalizations are based on the present incarnation of a category and then traced to some earlier period. For example, Petkova et al., (2014) examine the legitimacy of investing in the clean energy sector by counting historical references to the terms ``clean energy," ``green energy" and ``alternative energy" in media articles. In practice, categories emerge organically (Goldberg 2013) and may cycle through several incarnations of appropriate behavior before they settle on the perspicuously demarcated boundaries that appear to the present observer. More importantly, this formulation excludes the struggle between competing paradigms that characterizes the typical legitimation process (Colyvas and Powell 2007).

How then, do we attempt to understand the uncertainty caused by transitions in technology, taste and attention in an unbiased manner? Despite it's shortcomings, the literature on legitimacy provides several useful insights. Foremost, the frequency of  behavior is a meaningful yardstick of general acceptance (Hannan and Carroll, 1992). Second, the frequencies of language used in the media are representative of an unfolding legitimation process (Petkova et al., 2014; Rosa et al., 1999). Using these insights, I propose a general notion of environmental uncertainty based on changes in the frequency of language use. This can be accomplished without imposing ex post classifications by tracking language frequencies as they naturally occur--that is, the frequency distribution of descriptive words at various points in time. Changes in a frequency distribution from one period to the next is thus assumed to represent shifts in the importance of behavior described by the language. The volatility of these changes can capture disagreement about importance and hence uncertainty as previously described. 

\section{Data}

Talk about source, set up and cleaning

\section{Frequency Distributions}

Show change in top descriptive words over time

\section{Kullback-Leibler Divergence}

\begin{enumerate}
  \item Show how entropy divergence has dropped over time
  \item Show words associated with large jumps
\end{enumerate}

\section{Entropy and Exchange}
\begin{enumerate}
  \item Show patterns with trading volume
  \item Show statistical analysis of relation to trading volume
 \end{enumerate}


