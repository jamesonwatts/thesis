\chapter{Language Uncertainty and Firm Performance \label{signal}}


\begin{small}
\begin{quote}
``Orthodoxy is the death of knowledge, since the growth of knowledge depends entirely on the existence of disagreement" (Popper 1994 p. 34).
\end{quote}
\end{small}

Chapter \ref{law} introduced an information theoretic definition of knowledge heterogeneity based on the concept of entropy. As noted in section \ref{ent}, it can be tempting to conflate high levels of market entropy with the concept of uncertainty. This stems from the mathematical definition of high entropy as a property of probability distributions with little information (i.e. high disorder). However, information (under the proposed definition) refers to the \emph{differences} between market actors. When we can distinguish one actor from the next, we have some information about the market. When we can't, we have no information. Thus, it is meaningful differences (not similarity) that are the source of mathematical order in the market entropy construct. Whether this translates to uncertainty is entirely dependent on the interpretation of this property from the standpoint of some focal market actor.

Nonetheless, knowledge heterogeneity can be related to a particular type of environmental uncertainty in the current context. For instance, the uncertainty that results from disagreement over the importance (or appropriateness) of market behaviors (practices, products, etc.) as described in Chapter \ref{lang}. As noted earlier, this will naturally occur during transitions of technology, taste and attention if innovations are assumed to arise locally and diffuse gradually (Colyvas 2007; Rogers 2002). For a period following the innovation, the variance in knowledge and opinion of that innovation will be more pronounced and the extent or ultimate acceptance of the change, is temporarily unknowable. Thus, the uncertainty of consequence in the current context, is that stemming from an inability to anticipate future standards of quality.

But to what extent does this type of environmental uncertainty affect the behavior and performance of individual market participants? Given my focus on knowledge heterogeneity, we may be able to draw some insights from the substantial literature dealing with problems of information asymmetry. Consider the canonical argument from signaling theory, which states that under uncertainty (i.e. some asymmetry in knowledge), high-quality actors have an incentive to invest in signals of quality (Kirmani and Rao 2001). Perhaps they offer a warranty, and charge a slightly higher price. Lower quality competitors are unable to do the same because it is more costly for them to offer a warranty than it is for the high-quality producer. This is because their product (service, etc.) will fail more often. The signal is thus valued in the marketplace as a way to discern the price-quality tradeoff and, in equilibrium, the high-quality producers are more profitable (Kirmani and Rao 2001; Spence 1973).

In the above example, all rational high-quality producers in a field will invest in signals of quality. However, the typical signaling model assumes irreconcilable information assymmetries. In my context, I allow for variations in knowledge heterogeneity beyond that which is irreconcilable. Yet, this variation would have minimal impact on exchange behavior under the traditional signaling formulation. This is because reductions in information asymmetry lead to less investment in signaling, but not to changes in the set of partners chosen for exchange--less asymmetry reveals the same high-quality producers that were previously revealed via signals (Podolny 2005). However, this logic does not hold if acquisition of a signal is restricted by some criteria other than quality. If for instance, acquisition of a signal depends on position, legacy or capacity constraints, then it may not be available to all producers regardless of their quality.\footnote{See Podolny (2005) for a complete rendering of the status-based conception of Spence's (1973) signaling theory}

For example, in the biotechnology industry, alliances with the National Institute of Health (NIH) can act as a signal of quality for fledgeling firms (Baum et al. 2000; Podolny 2001; Shane and Stuart 2002). However, the NIH has limited resources and cannot form alliances with all the high-quality biotech firms that exist. Thus, many equally (or more) qualified firms may enter the field, but lack the ability to signal quality through an alliance with the NIH. In cases where access to signals of quality are restricted, variations in knowledge heterogeneity will have a more profound impact on behavior. This is because an actor's quality becomes more or less apparent independent of their market signal. Provided some actors differ only in their signals and not in underlying quality, then the advantage realized by actors in possession of a signal will shift to others when quality uncertainty decreases.

In the sections that follow I expand on the above arguments to investigate firm performance during the biotechnology industry's tumultuous evolution. To do so, I've assembled data from four distinct sources. I use the dataset collected by Powell and his colleagues (see Powell et al. 2005; Powell, Koput and Smith-Doerr 1996), extended through 2004 as the source for information on certain firm attributes and their alliances. Stock market data and firm financials were collected from the Wharton Research Data Service (CRSP/Compustat merged). Data on patents was collected from Commercialization Research on Innovation and Entrepreneurship (CRIE) database hosted by the University of Arizona \footnote{Additional data on patents used in several robustness tests was provided by Powell and colleagues. However, this data only extends to the year 2000.}, and textual data was collected and prepared as described in Chapter \ref{lang}. 

A measure of language uncertainty is constructed from the volatility of language consistency and is shown to moderate the advantages derived from centrality in a network of strategic alliances. Whereas language uncertainty positively moderates a central actor's ability to acquire new resources, it negatively moderates that actor's productive output. These results are explained using theory which describes network position as both symbolic and substantive in its influence over the behavior and performance of market actors.

\section{Structural Signals of Quality \label{sig_qual}}

In the classic model proposed by Spence (1973), a useful market signal must 1) be at least partially manipulable by the actor and, 2) the difficulty in obtaining the signal must be inversely correlated with the actor's quality level. Under this model, it is assumed that access to indicators of quality are equally available to all actors desiring to signal high quality. This assumption is valid in markets where the acquisition of signals is regulated or takes place through arms-length transactions. For instance, high-quality firms may spend more to advertise their products (Kirmani and Rao 2001). Presumably, the purchase of advertising space is based entirely on one's ability to pay for it. Indeed, the bidding and allocation of advertising space is often fully mediated by computers, thus removing many of the biases that might otherwise permeate human-to-human interactions. And yet, many markets are \emph{socially} mediated and display biases based on gender (Koput and Gutek 2010), location (Owen-Smith and Powell 2004) and status (Kim and King 2014) to name just a few. 

An alternative model--inspired by Merton (1963) and articulated by Podolny (1993; 2005)--argues that the signaling value of social position (i.e. structure) can affect exchange behavior independent of underlying quality. That is, given some baseline level of quality, benefits to those in a more favorable position will accrue at a faster rate than those in a less favorable position (Lynn, Podolny and Tao 2009). This rendering is mathematically consistent with the Spencian view if 1) the indicator of quality is not available to all actors desiring to signal high quality and, 2) the cost of production at a given level of quality is inversely associated with possession of the indicator. For instance, two equally qualified candidates may apply for a postdoc position at Stanford. The individual who receives the position benefits from a lower cost of producing research. This occurs because they have access to needed funding and are more likely to get invitations to present their research at other quality institutions, and thus receive useful feedback (Podolny 2005).

A steady stream of empirical research has advanced the social-structural view of market signals by documenting largely positive returns to advantageous structural positions (Benjamin and Podolny, 1999; Bothner, Kim, and Lee, 2014; Kim and King 2014; Podolny, 1993; Powell et al. 1996). However, recent theoretical arguments have criticized this work for its narrow focus on structural position as a production advantage (Washington and Zajac, 2005; Jensen, Kim, and Kim, 2011). In a competing formulation, acquisition of `positional goods' (e.g. Hirsch 1977) have been postulated to accumulate returns based on purely symbolic effects, {\it over and above} any effect that the advantageous position has on cost. The scarcity of such positions renders their occupation the equivalent of a status symbol. Other market participants can thus assert their prominence via conspicuous affiliation with these well-positioned actors (Bourdieu 1984; Veblen 1899). 

Malter (2014) confronts the challenge posed by these competing arguments by studying whether a wine's status ranking affects the price at which it is sold. He uses the grand cru classification of the ch\^ateaux of the M\'edoc, which sorted 61 producers into five grands crus class\'es in 1855 as a basis for a presumed status ordering and argues that underlying quality can be objectively distinguished from status via each wine's publicly available ratings (published online). However, unlike prior studies, he controls for the additional statistical bias of reverse causality by noting that the status ranking has not changed in 150 years (and doesn't change over the period of study). He concludes that the quality-independent price premiums sustained by high-status wineries are evidence of a purely symbolic (i.e. conspicuous) purchase motive.

Despite a laudable research design, the arguments of Malter (2014) entail an unnecessarily confined perspective on quality observability. The claim--as it stands--is that consumers prefer a high-status wine despite knowledge of a less costly alternative of equal (or greater) objective quality. This preference is based on the consumer's ability to conspicuously consume the status symbol in front of an appreciative audience. However, quality, in the Malter (2014) study is assumed to be fully (and objectively) observed via the wine's online rating. For the conspicuous consumption argument to truly hold--that is, for there to exist a purely symbolic utility--the audience in front of which the wine is consumed would also need to know of the less costly, but objectively equivalent alternatives. That is, all audience members would need to know the online ranking at the point at which they observe the act of conspicuous consumption.

While it is possible that some audience members are aware of the equivalent quality, lower cost alternatives, a more realistic assumption is that many in the audience still rely on the grand cru classification as a signal of high quality. Consider the likely consumption scenarios. The consumer drinks the wine on her own then tells an audience about the experience, she shares the wine with an audience or, she uses it as a gift to an audience member. Symbolic utility may arise from a demonstration of the ability to pay a higher price (or a disregard for money); however, it is just as likely that it accrues as the result of information asymmetry once removed. The original purchaser is willing to pay a price premium despite knowledge of equal-quality alternatives because the audience still relies on the original signal of quality--they are aware of the classification, but unaware of the wine's online ratings. 

One could think of a similar argument in the context of a marketing channel. A retailer is willing to pay a premium on the transfer (wholesale) price of a brand name product despite knowledge of an equivalent generic because the retailer's customers still value the brand as a signal of quality. Similarly, a firm may seek an alliance with a high status alter despite private knowledge of equivalent (lower status) alternatives because others in the field are as of yet, unaware of the equivalence and still rely on the alter's status as a proxy for the value of the relationship.  

Without precluding the existence of purely symbolic advantage, this discussion reveals an interesting fact about structural market signals. One need not prove information asymmetry in a particular relationship to make claims about their value as market signals. Instead, it may be sufficient to claim that knowledge heterogeneity exists in the broader market context. That is, benefits can accrue to actors in a favorable position even if bestowed by those with full knowledge of underlying quality, provided some information asymmetries still exist elsewhere in the field. 

\section{Pipes or Prisms?}

Despite the preponderance of research documenting structural advantage (e.g. Benjamin and Podolny, 1999; Bothner, Kim, and Lee, 2014; Kim and King 2014; Podolny, 1993), the mechanisms undergirding such advantage are often left underdeveloped (Jensen, Kim, and Kim, 2011; Podolny 2001). As noted above, this was the primary motivation for the Malter (2014) study; however, this lack of definition is particularly evident in research on networks where `structure' is conceptualized as ``...enduring patterns of relationships among actors--be they individuals, cliques, groups, or organizations" (Powell et al. 1999 p. 2). Here, the cost advantages of a favorable position can accrue on both a symbolic \emph{and} a substantive basis (see e.g. Podolny 2001), which complicates the process of theory identification.

Podolny (2001) illustrates these distinct types of advantage via the analogy of networks as ``pipes" and ``prisms," of the market. When networks are viewed as the ``pipes" or ``plumbing" of the market, they serve as channels or conduits through which resources and information are exchanged between actors. By positioning oneself at a crucial juncture, certain actors find advantage in the form of better and earlier access to information (Powell et al. 1996) or as brokers between more distant exchange partners (Burt 1992).

However, when viewed as informational cues, network affiliations become the ``prisms" through which assessments of quality are performed (i.e. they provide a symbolic point of reference). For instance, Baum and Oliver (1992) show that daycare centers are viewed more favorably by consumers when they establish ties to important organizations in the community like governmental agencies or church groups. When viewed as an information cue, certain network structures can provide a signaling advantage under quality uncertainty as argued in Section \ref{sig_qual}.

Distinguishing between these forms of advantage is often a matter of context; however, a few generalizations are apparent. For instance, any substantive advantage is based on a focal actor's ability to use their position to translate the information or resources flowing across existing network ties into productive output. Thus, the magnitude of advantage should be proportional to the quality of the resources flowing across these ties. In contrast, symbolic advantage stems from the fact that \emph{new} resources flow to a focal actor that may not have otherwise. In this case, the magnitude of advantage depends on the degree of uncertainty about the quality of the outputs the focal actor will produce with these resources in the future as previously discussed.

Both of these mechanisms are motivated in the well known work of Powell et al., (1996) and the authors' followup study in Powell et al., (1999). They show that firms located at the center of a network of strategic alliances tend to outperform their peers on a number of dimensions. Certain performance outcomes are based on the acquisition of new resources, while others have to do with the transformation of extant resources into productive output. However, both the substantive and symbolic advantages of network centrality are given equal empirical footing. The conditions under which a given type of performance is skewed towards one source of structural advantage or the other, is left largely untested. 

I propose that language uncertainty can be used as a way to identify the influence of these distinct mechanisms. If the consistency of language can proxy for field-wide shared understanding as discussed in Chapter \ref{lang}, then \emph{inconsistency} in language could adversely affect the value of information garnered from extant network ties. This in turn, makes it more difficult to translate that information into productive output. However, a lack of field-level shared understanding may also increase uncertainty regarding the future standards of quality. This could actually intensify the symbolic significance of a central network position for those looking to invest new (or further) resources in an industry. 

For centrally located firms, these countervailing forces lead to an unusual tension. Under high language uncertainty, the quality signaled by a firm's network can bring it additional resources--including additional network ties. However, it also becomes more difficult to translate these resources into productive output. Thus, language uncertainty is expected to positively moderate any symbolic structural advantage while negatively moderating the substantive advantage.

Using the dataset from Powell et al., (1999) extended and updated by five years, I test the moderating role of language uncertainty on the performance advantages of network centrality. I take the work of Powell and colleagues as my point of departure for the empirical portion of this chapter and attempt to reproduce their original results before introducing the language uncertainty moderator. In the next section I briefly review my empirical context and then describe my data and measures. This is followed by a description of my methods and related assumptions. I conclude with a discussion of the results and their significance.

\section{Context}

My research setting is the biotechnology industry during the period 1993-2003. This time period is important for two reasons. First, it predates widespread use of the Internet as a primary source for information on the Biotechnology industry. This increases confidence in the reliability of my language measure developed in Chapter \ref{lang} since few alternatives to the Bioworld trade journal were available (Wolfe 2001). Second, it represents a period in the industry with sparse analyst coverage and unreliable financial data--there were few individuals capable of evaluating the science, much less its commercial value (Powell et al. 2005; Wolfe 2001). These features of the industry amplified the importance of strategic alliances as an indicator of firm potential (Podolny 2001; Shane and Stuart 2002). The fact that significant investment has occurred in this context confirms the importance of network models in information and investment. Moreover, the data I describe below are based on formal contractual agreements and not informal ties, handshake deals, or social embedding, hence they afford a strict test of whether network relationships influence financial outcomes.  

The science underlying the field of biotechnology had its origins in university laboratories. These promising discoveries were initially exploited by a handful of science-based start-up firms founded in the mid-to-late 1970s. The year 1980 marked a sea change, with the U.S. Supreme Court ruling in the Diamond v. Chakrabaty case that genetically engineered life forms were patentable. Congress passed the Bayh-Dole Act in the same year, which allowed universities, nonprofit research institutes, and small businesses to retain the intellectual property rights to discoveries funded by federal research grants (Mowery, Sampat, and Zedonis, 2001). And Genentech--which along with Cetus was then the most visible biotech company--had its initial public offering, drawing great interest on Wall Street, with a single day stock price run up exceeding any previous one-day jump. Over the next two decades, hundreds of small, science-based biotech firms were founded, mostly in the United States but more recently in Canada, Australia, Britain, and Europe.

In fields such as biotech, where knowledge is advancing rapidly and the sources of knowledge are widely dispersed, organizations enter into an array of alliances to gain different competencies and knowledge (Powell, et al., 1996). In so doing, firms develop portfolios of relationships that permit access both to developments in science and skill in bringing new products to markets. Thus, the field is not only multi-disciplinary, it is multi-institutional as well. In addition to research universities and both start-up and established firms, government agencies, nonprofit research institutes, and leading hospitals have played key roles in conducting and funding research, while venture capitalists and law firms have played essential parts as talent scouts, advisors, consultants, and financiers (Gilson and Black 1998; Lerner and Merges 1998), and chemical, pharmaceutical, health-care, and conglomerate corporations have sought to bring the science to consumers.  

\section{Data Sources}

As a starting point, I use the dataset collected by Powell and his colleagues (see e.g. Powell et al. 2005), extended through the end of 2003. The dataset on firm demographics and inter-organizational agreements has been described in detail elsewhere, but a few remarks are still warranted. The database focuses on dedicated human biotech firms (DBFs), omitting companies involved in veterinary and agricultural biotech (which draw on different scientific capabilities and operate in a much different regulatory climate).  The reference source for information on a firm's ownership, financial history, formal contractual linkages to collaborators, products, and current research was the proprietary industry directory BioScan. Firm characteristics reported in BioScan include founding data, employment levels, financial history, and, for firms that exit, whether they were acquired or failed. The data on alliances cover the time frame and purpose of the relationship. Large pharmaceutical corporations, health care companies, hospitals, universities, and research institutes enter as 2-mode partners.

Network measures are constructed based on the pattern of strategic alliances between firms. Alliance formation is tracked as far back as 1967 to account for contracts already active in 1993. Firms in my sample varied widely in their number of active alliances within a given year--several DBFs maintained only a few alliances while others were active in more than 100. On average, firms grew their strategic alliance portfolios from just over seven alliances in 1993 to more than eleven by the year 1999. The years 2000 through 2003 saw a modest decline in the number of overall alliances as firms succumbed to the financial pressures characteristic of that time period. 

Language data was collected and prepared as described in Chapter \ref{lang}. Combining this with the DBF database results in a sample that covers 655 firms, of which 402 were in existence in 1993 and 478 in 2003. In this time-frame, some firms were created and entered the database, and others exited, due to failure, departure from the industry, or acquisition. The number peaked in 2001 with 541 firms.

Financial data was obtained from the Wharton Research Data Service (CRSP/Compustat merged), a widely used electronic data service. However, use of this data constrains the size of my panel in several models since it only tracks data on publicly traded firms. Of the 655 firms identified in Bioscan, 153 were publicly traded in 1993 and 252 in 2003 with a peak of 293 in 2001. Among those listed on an exchange, many were only public for a single year, which would limit any ability to track changes over time. A few others are excluded because of missing data, which occurred most often for international firms. 

Patent data was collected from two separate sources. The primary source originates with the dataset collected by Powell and colleagues for their 1999 expansion of the Powell et al., (1996) research. This data comes from CASSIS, which is a government document that was made available by the U.S. Patent and Trademark Office before data was available over the Internet. This data is supplemented with patent counts from a search done through the Commercialization Research on Innovation and Entrepreneurship (CRIE) database hosted by the University of Arizona. This database uses a matching algorithm to associate a company's unique identifier (such as `permno') with the data published online by the U.S. government. This is also limited to publicly traded companies.

Most models are run with $N>250$ and approximately 1,700 firm-year observations. The most expansive model is unrestricted by whether a firm is publicly traded and covers essentially the entire database. The most restrictive panel has $N=173$ and 823 firm-year observations. However, this occurs with the additional restriction of a minimum 5 years of stock market data. The justification for this particular restriction is explained in the Measures section below. All models are shown to be robust to a variety of selection biases in Appendix B.

\section{Measures.}

For any given DBF, a network profile consists of the number of ties (alliances) it has for each of four core types of business activity--research, financing, licensing and commercialization. The number of research ties is an important outcome variable as well as a component of a firm's network profile and so it is treated separately (both conceptually and empirically). The range of activities that a firm is engaged in during 

Since the primary goal of this Chapter is to extend the theory and analysis of Powell et al., (1999), I adhere in a strict sense to their measure of network centrality. 


\section{Model}
The model of Powell et al. (1999) is one in which network centrality plays a substantial role in determining firm performance. This can be seen clearly from Figure \ref{omodel} reproduced here by permission of the original authors. However, this visual representation illustrates results from a 3-stage regression and many of the mechanisms driving these results are hidden in the one and two stage results.  For instance, It plays a symbolic role in attracting the new resources that lead to sales and growth, and a substantive role facilitating the translation of research and development collaborations into new patents. Figure \ref{model} shows the 

\begin{figure}
\begin{center}
\includegraphics[scale=.60]{../figures/omodel.png}
\caption[Powell et al., (1999) Model]{Theoretical model from Powell et al., (1999). This is an extension of the authors' original (1996) model which can be seen in the shadowed components above. This is reproduced here by permission of the authors.\label{omodel}}
\end{center}
\end{figure}


\begin{figure}
\begin{center}
\includegraphics[scale=.75]{../figures/model.png}
\caption[Refined Theoretical Model]{Refined theoretical model with interactions and clarification of symbolic vs. substantive effects. Only the tested relationships are included.\label{model}}
\end{center}
\end{figure}

Fore each performance measure, I first test the relationships using the same methods as those used in the original paper and then compare those results with those obtained from newer methods that control for additional sources of potential bias.

\section{Results}
\section{Discussion}