\chapter{Sample Second Chapter\label{chap2}}

\section{Remembering to use non-breaking spaces}

The tilde character can be used to mark a non-breaking space.  \LaTeX\
does a great job of lining words up, putting in line breaks where
appropriate, and making the text flow properly.  However, there are
some cases where you really don't want \LaTeX\ to toss in a carriage
return.  The places that spring to mind are between numbers and
their units, and between Figure and Table references and their
values.  So for example, you don't want something like this in your
document:

\begin{quote}
	Blah blah blah blah blah blah blah blah blah is 10\\
	m from foo to bar.
\end{quote}

So always use the tilde when you use numbers with units, like
9.8~m~s$^{-2}$, and with Figure and Table references, like
Figure~\ref{samplefig} and Table~\ref{sampletable}.  This way \LaTeX\
will never put a line break between those elements.


\section{Tables\label{tables}}

On the next page is a sample table, placed on the page by itself.
Again, this table is pretty small, so it could probably just be
placed on a page with text.  Messing with tables can be quite
grueling, so find yourself a good book.  Sometimes tables can be
wider than they are tall, and you may need to rotate the table by
$90^{\circ}$ to make it fit better on a page by itself.  To do that
you can use the lscape package.  To use it, you need to do two
things, uncomment the usepackage statement in the main dissertation
file, and wrap the table commands in a begin and end landscape
command (like I have commented out in this file), and that table
will be properly rotated.  Go ahead, try it out.

% \begin{landscape}
\begin{table}[p!]
\begin{center}
\caption[Short table caption for LOT]{Sample table caption (to appear with the actual table). \label{sampletable}}
\vspace{0.3in}
\begin{tabular}{ccc}
\hline 
\hline
Col A & Col B & Col C \\
\hline
1 & 2 & 3 \\
4 & 5 & 6\\
\hline
\end{tabular}
\end{center}
\end{table}
% \end{landscape}

Note that the \verb=\caption= command can have a short and a long version
inside a table environment, just like inside a figure environment (see \ref{graphics}).
