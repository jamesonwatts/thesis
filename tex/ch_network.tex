\chapter{Communicability and Network Advantage \label{signal}}


\begin{small}
\begin{quote}
``Many products have failed to gain acceptance from particular groups of customers, despite the undeniable advantages which they offered, because they were not compatible with one of the sets of values by which those customers appraised them." (Loasby 1999 p. 42) 
\end{quote}
\end{small}

Chapter \ref{law} introduced an information theoretic definition of knowledge heterogeneity based on the concept of entropy. As noted in section \ref{ent}, it can be tempting to conflate high levels of market entropy with the concept of uncertainty. This stems from the mathematical definition of high entropy as a state with little information (i.e. high disorder). However, information (under the proposed definition) refers to the \emph{differences} between market actors. When we can distinguish one actor from the next, we have some information about the market. When we can't, we have no information. Thus, it is meaningful differences (not similarity) that are the source of mathematical order in the market entropy construct. 

Nonetheless, knowledge heterogeneity may describe a particular type of environmental uncertainty. For instance, the uncertainty that results from disagreement over the importance (or appropriateness) of market behaviors (practices, products, etc.) as described in Chapter \ref{lang}. As noted earlier, this will naturally occur during transitions of technology, taste and attention if innovations are assumed to arise locally and diffuse gradually (Colyvas 2007; Rogers 2002). For a period following the innovation, the variance in knowledge and opinion of that innovation will be more pronounced and the extent or ultimate acceptance of the change, is temporarily unknowable. Thus, the uncertainty of consequence in the current context, is that stemming from an inability to anticipate the future standards of quality.

But to what extent does this type of environmental uncertainty affect the exchange behavior of individual market participants? Given that I am focused on knowledge heterogeneity in a field, we may be able to draw some insights from the literature on information asymmetry. Consider the primary argument from signaling theory, which states that under uncertainty (i.e. some asymmetry in knowledge), high-quality actors have an incentive to invest in signals of quality (Kirmani and Rao 2001). Perhaps they offer a warranty, and charge a slightly higher price. Lower quality competitors are unable to do the same because it is more costly for them to offer a warranty than it is for the high-quality producer. This is because their product (service, etc.) will fail more often. The signal is thus valued in the marketplace as a way to discern the price-quality tradeoff and, in equilibrium, the high-quality producers are more profitable (Kirmani and Rao 2001; Spence 1973).


\section{Context}
\textbf{Textual data.}\\
\textbf{Network data.}\\
\textbf{Measures.}\\
\section{Model-free Evidence}
\section{Statistical Model}
\section{Results}
\section{Discussion}